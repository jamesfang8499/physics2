\chapter{物态变化}
\section{教学要求}
本章讲解的知识,是对初中学过的物态变化知识的扩展和加深,与生产、科研和日常生活实际有着密切的联系。

教材对熔解、凝固、汽化等现象,和熔解热、汽化热等概念,从分子运动论的观点和能量的观点作了定性的解释,是为了使学生对现象和概念理解得深刻一些;这也有利于培养学生的抽象思维能力。这一章的现象多、概念多,讲述时除了加强实验、增加感性知识外,还要多举生活中的实例,帮助学生理解。还要注意弄清楚某些具有共同特点的概念间的区别(如熔解热与汽化热)。

本章的教学可分为三个单元:第一单元,包括第一、二节,讲熔解和凝固的知识,第二单元,包括第三节到第七节,讲解蒸发、沸腾及饱和汽和气体的液化的知识,第三单元,包括第八、九节,讲解空气的湿度和露点。

下面对这一章的教学内容作一些具体说明。

晶体和非晶体在熔解时的区别初中已学过,只要求作简单的复习,课本对此从分子结构不同所作的解释,以及对熔
解热用分子运动论和能量转化的知识所作的说明,可以使学生从理论上加深对这些现象的认识,同时使他们了解理论的应用,从而巩固理论知识。

饱和汽的知识是理解液体的沸腾、气体的液化、空气的湿度、露点等许多现象的基础,需要重点讲解,它又是学生新接触的理论性较强的内容,也是教学的难点,理解饱和汽的知识,关键在于讲好动态平衡的概念。

液体沸腾的条件和沸点跟压强的关系,课本是通过实验观察得出结论的。至于为什么当饱和汽压等于外界压强时液体才沸腾,这个问题在中学阶段很难讲清楚,教材中也没有进一步分析这个问题,希望在教学中注意掌握,不做过高的要求。

课本对汽化热也用分子运动论和能量转化的知识作了说明,液体汽化时用于克服外界压强所做的功不能忽略,沸点变化时汽化热差别较大,这些与熔解热的不同点,要向学生指明。

空气的湿度和露点,在生产和生活实际中常常用到,课本讲解这些知识,就是为了给学生理解一些实际现象打下一个初步基础。

根据以上分析,本章的教学要求是:
\begin{enumerate}
\item 了解晶体和非晶体熔解和凝固时的不同,掌握熔解热的概念。
\item 知道蒸发和沸腾的区别,了解沸腾的条件及沸点跟压强的关系,掌握汽化热的概念。
\item 知道什么是饱和汽,理解动态平衡的概念.了解饱和汽的压强(或密度)跟温度有关,跟体积无关,知道气体液化的方法,了解临界温度的概念。
\item 理解绝对湿度、相对湿度和露点等概念.
\end{enumerate}

\section{教学建议}
\subsection{第一单元}
\subsubsection{熔解和凝固} 

可以先让同学观察萘的熔解和凝固曲线(图5.1和图5.2),思考以下几个问题:
\begin{enumerate}
\item 两个图各是什么曲线?    \item 曲线的哪一段表示物质的固态、液态或固液共存的状态?    \item 曲线的哪一段是熔解和凝固的过程?    \item 熔点和凝固点各是多少度?它们有何关系?
\end{enumerate}
这样就可以把熔解和凝固的过程以及熔点的概念搞清楚了。
\begin{figure}[htp]\centering
    \begin{minipage}[t]{0.31\textwidth}
    \centering
    \includegraphics[scale=.8]{fig/5-1.png}
    \caption{}
    \end{minipage}
    \begin{minipage}[t]{0.31\textwidth}
    \centering
    \includegraphics[scale=.8]{fig/5-2.png}
    \caption{}
    \end{minipage}
    \begin{minipage}[t]{0.31\textwidth}
        \centering
        \includegraphics[scale=.8]{fig/5-3.png}
        \caption{}
        \end{minipage}
    \end{figure}

同样,通过对比萘的熔解曲线和松香的熔解曲线(图5.1和图5.3),可以使学生认识晶体和非晶体熔解过程的不同特点。

对晶体的熔解和凝固过程,运用分子运动论从晶体和液体的微观结构进行解释时,可以先结合熔解曲线对熔解过程作出示范分析,对凝固过程则可让学生自己进行分析说明。

要在讲清多数物质熔解时体积变大、少数物质熔解时体积变小的前提下,讲解物质的熔点与压强的关系,可以先引导学生看课本上各种物质的熔点表,并使他们懂得这些熔点是在一标准大气压下测得的。进而讨论熔点跟压强的关系,还须指出,熔解时物体的体积无论是增大还是减小,都需要吸收热量,用来克服分子力做功,破坏空间点阵,掺杂对熔点的影响主要通过冰、盐混合物熔点降低等实例说明。

\subsubsection{熔解热}

要从熔解过程中能量转换的角度说明熔解时吸收热量,用于克服分子力做功,破坏晶体的空间点阵,增加物体的分子势能。同时,还要说明,由于不同晶体的空间点阵不同,单位质量不同的物质熔解时吸收的热量也不同,为表征物质的这一性质,引入熔解热的概念。可以把熔解热跟比热、热量等概念从物理意义、定义、单位等方面加以对
比区别,加深对熔解热的认识,还应指出,熔解热与凝固时单位质量的物质放出的热量相等,是能量守恒定律的必然结果。

课本上的测定熔解热的方法,需要用到热量的计算、热平衡方程和量热器的构造等初中已学过的知识,需要适当的复习。为了具体地掌握测量的方法,可以一边进行实验一边讲解,把测量熔解热的方法具体化,结合熔解热的测定,还要讲清分析和解决包含物态变化过程在内的热平衡问题的基本思路:首先要明确研究对象,知道参加热量交换的物体是哪些?它们各自的初状态和应该达到的末状态的物态和状态参量(温度、压强、体积)是怎样的?状态变化的过程经历了哪些阶段?中间状态的物态和状态参量又是怎样的?然后,分别列出所有放热物体放出热量和所有吸热物体吸收热量的数学表达式。最后,由热平衡方程列式求解,在不知道末状态物质处于什么物态时,需要先判断末状态是什么物态。可以熔点温度为界线,结合熔解热进行热量的估算,具体判定。比如水与冰混合,可以把水温降低到0℃放出热量跟冰上升到$0^{\circ}{\rm C}$和完全熔解所需的热量进行比较,来判断末状态是水、是冰还是冰水混合物。

\subsection{第二单元}
\subsubsection{蒸发}

蒸发现象、影响蒸发快慢的因素、蒸发的致冷作用等,在初中学过,可以通过举例复习、巩固这些知识,教材中增加的内容是:用分子运动论解释蒸发现象、介绍蒸发
致冷的应用实例,这些学生不难理解。可以让他们自己阅读,然后组织讨论,把学习搞得生动活泼一些。

\subsubsection{饱和汽与饱和汽压}

需要帮助学生复习一些已有的知识为学习新课做准备.可以通过提问,复习以下问题:
\begin{enumerate}
\item 描述气体状态的三个参量(温度、体积、压强)的意义及微观解释是怎样的?
\item 影响蒸发快慢的因素是什么,并用分子运动论加以说明。
\end{enumerate}

正确理解动态平衡的意义是掌握饱和汽概念的关键。可引导学生通过课本图5.2从微观的角度想象,密闭容器内分子逸出液面和返回液面的运动状况,着重让学生理解这两种运动宏观上达到平衡时(即汽体的密度不变,液体不再减少),并非意味着分子运动的停止,而是单位时间内逸出液面的分子数与回到液面内的分子数相等,即处于动态平衡。在此基础上讲清什么是饱和汽和未饱和汽,以及“在一定温度下,未饱和汽密度小于饱和汽密度”的道理。

动态平衡是有条件的,对此要讲解清楚。由于外界条件变化的影响,原来的动态平衡状态被破坏,经过一段时间才能达到新的平衡。比如,温度升高时,分子平均动能增大,单位时间内逸出液面的分子数增多。于是原有的动态平衡状况被破坏;空间汽分子密度逐渐增大,导致单位时间内返回液面的分子数增多,从而达到新的条件下的动态平衡。由此得出,饱和汽的密度随着温度的升高而增大,掌握了动态平衡状态变化的条件,才能更好地理解饱和汽的性质。

饱和汽压,从微观上讲仍然决定于分子密度和分子的平均速率。讲清这一点,能消除对饱和汽压的陌生感,帮助学生
认识饱和汽压的实质。要做好测定饱和汽压值的实验,这不仅能使学生对饱和汽压获得感性认识,也能使他们了解一种测量饱和汽压的方法。

研究饱和汽压跟温度的关系和跟体积的关系,要先从演示实验得出结果,再从理论上加以说明,使学生对结论理解得更确切、深刻些,课本上的实验也可以用其他实验代替,但对课本上的实验也要讲一讲。

在得出“饱和汽压随温度的升高而增大”的结论后,还要进一步讲解饱和气压随温度怎样变化,其关系如课本图5.4所示,可以跟理想气体的等容线作对比,一定质量的理想气体在体积不变时,其压强与绝对温度成正比,从微观上讲,是质量一定、体积一定,因而分子密度未变;但温度升高,分子平均动能变大,平均速率变大,导致压强增大,对饱和汽来说,温度升高时,不仅分子平均动能变大,分子平均速率变大;同时液面进入空间的分子数增多,分子的密度也增大,决定压强的这两个微观因素都变大,这就使饱和气压的值增大得更多。

在讲解饱和汽压跟体积变化的关系时,要注意强调前提条件:温度不变。再从饱和汽密度与温度的关系和动态平衡状态的条件,讲清体积变化时饱和汽压值不变的道理,由于温度不变,分子平均动能不变,分子平均速率不变,设若汽的体积变大,分子密度变小,小于这一温度下饱和汽应有的密度,成为未饱和汽,破坏了原有的动态平衡,液体就会继续蒸发,使汽恢复到原有的分子密度,成为饱和汽,建立起新的动态平衡。因此,决定压强的两个微观因素都没有变化,因而压强不变。这里还应结合温度一定,饱和汽体积变小的过程,讲
清一部分汽会凝结成液态达到减小体积后的新的动态平衡。

要注意归纳饱和汽的特点,说明它与理想气体不同,不能用理想气体定律来解决饱和汽的问题。

\subsubsection{沸腾}

用课本图5.6的实验现象,讲解什么是沸腾以及沸腾的过程,要把它与蒸发对比,让学生认识这两种汽化现象的区别,由观察现象让学生了解:沸腾前,液体内气泡在上升过程中体积变小;沸腾时,液体内气泡在上升过程中体积变大。对这一现象不要作过细分析,为了利于学生理解沸腾的条件,可以讲到这样的程度:沸腾前,容器内壁吸附的空气在器壁和器底形成小气泡,由于周围的水向气泡内蒸发,气泡中除了空气外还有水的饱和汽。当器底的气泡受到液体的浮力上升时,因为液体底部的温度高于上部,气泡在上升过程中温度下降,气泡内部的饱和汽不断液化,气泡的体积不断变小。当液体内各部分的温度都达到某一温度,气泡内的饱和汽压值等于外部大气压强时,气泡在上升过程中体积不再减小,而且由于周围液体不断向泡内蒸发,体积还会继续变大。到达液面时破裂放出饱和汽,液体就沸腾了,这样就可以使学生理解液体沸腾的条件是它的饱和汽压等于外界的压强。

关于沸点,除了引导学生理解课本对沸点的解释外,要强调跟外界压强相等的饱和汽压对应的温度,就是液体的沸点。这样才能帮助学生理解沸点随外界压强变化的关系,对此,除了道理上要讲清楚,还要用演示实验来验证,再用这一关系去说明一些现象,如离地面越高沸点越低,高压锅、蒸汽锅炉用增大压强的办法来提高沸点等。

\subsubsection{汽化热} 

先以蒸发时致冷、沸腾时液体继续吸热而温
度不变的现象,讲解液体汽化时要从周围吸收热量,还要让学生从能量方面了解,固体熔解时,由于体积变化较小,吸收的热量主要用来克服分子间的引力做功;而液体汽化时,由于体积明显增大,吸收的热量,一部分用来克服分子间的引力做功,另一部分用来克服外界压强做功。这是熔解过程和汽化过程的不同之处。

关于汽化热,可以对照熔解热,讲清它的定义,公式和单位,再要求学生阅读课本第101页的两个表格,思考这两个表格各自说明什么问题,怎样用分子运动论作出解释?然后通过讨论,了解不同物质在同一压强下汽化热不同,同种物质在不同温度下汽化热不同的道理。比如,不同物质分子间的距离和作用力的性质、大小不同,汽化时分子逸出液体所做的功不同,因而汽化热不同,再用分子运动论和能量转化的知识,说明在某一温度下气体凝结为液体放出的热量,与同一温度下汽化过程中吸收的热量是相等的。

关于汽化热的测定,要讲好或做好课本图5.8所示的实验。要使学生了解测定汽化热的装置及各部分的作用,实验的原理和方法,引导学生用所测出的各个物理量写出计算汽化热的表达式。当然,也可以把要讲解的内容以问题的形式提出来,让学生结合实物和实验过程展开讨论,得出结论。这样,既能加深对汽化热的理解,又能帮助学生学会应用汽化热等知识,分析解决包括汽化和液化过程在内的热平衡问题。分析解决这一问题的基本思路,可以参照解决熔解热问题的有关步骤,在分析参加热交换的各个物体的物态和温度时,要把汽化或凝结过程中汽液共存的状态考虑在内,在计算热量
时,不要漏掉计算汽化或凝结时吸收或放出的热量。

\subsubsection{气体的液化}

首先要引导学生回忆课本图5.5所示的实验:管内的饱和汽在温度不变、体积变小时,要凝为液滴,管内积存的液体增多,让学生体会到,气体液化的关键是把未饱和汽变为饱和汽。

关于把未饱和汽变为饱和汽的方法,可以先让学生思考两个问题:
\begin{enumerate}
    \item 相同温度下,饱和汽与未饱和汽的密度有何不同?
    \item 在温度不变时可否采取改变体积的办法来使未饱和汽转变为饱和汽?
\end{enumerate}
组织学生讨论得出:在温度不变时,减小未饱和汽的体积,使它的密度增大到这个温度下饱和汽的密度,未饱和汽就变成饱和汽了,再让学生思考以下问题:
\begin{enumerate}
\item 不同温度的饱和汽的密度相同吗?饱和汽的密度与温度有何关系?    \item 把高温下的未饱和汽的温度降低,能使它变为某个低温下的饱和汽吗?
\end{enumerate}
组织学生讨论得出:降低温度可以使未饱和汽变为饱和汽,在这过程中,容器内汽的密度没有发生变化,只是高温下的未饱和汽密度等于某个低温下的饱和汽密度,概括起来,把未饱和汽变为饱和汽的方法:一是减小体积(增大压强),二是降低温度。

使饱和汽凝结为液体,仍可采取减小体积和降低温度的方法,那么,只采用增加压强(减小体积)的办法能否使所有的气体都液化呢?用历史事实说明是不可能的,由此引入临界温度的概念,要强调在这个温度以上,物质只能处于气态,不能单纯用增大压强的方法来使它液化,临界温度是每种气体都具有的一个特殊温度,它的物理含义是:临界温度是物质以液态存在的最高温度,还可以通过临界管实验观察乙醚、水
等物质的临界状态,再通过物质的临界温度的表,弄懂氢、氧、氮等气体在历史上为什么会被当成“永久气体”的原因。

要重视液态气体和低温技术应用的介绍,讲清楚一、两个典型例子,以使学生眼界开阔、思维活跃。

\subsection{第三单元}
这一单元围绕湿度和湿度的测定,讲解了绝对湿度,相对湿度、露点的概念,介绍了几种湿度计的原理和使用方法,是已学基础知识的应用,与实际有紧密的联系。由于又引入了一些新概念,教师要注意引导启发学生用已学知识去理解新知识。

\subsubsection{空气的湿度}

要联系生活事例,说明空气的干湿程度是经常变化的,再讲解绝对湿度的定义,绝对湿度的初始定义-空气中所含水蒸气的密度,即单位体积中所含水蒸气的质量,学生是可以接受的。过渡到“空气中所含水蒸气的压强,叫空气的绝对湿度”时,须引导学生从决定气体压强的两个微观因素:分子的密度和分子运动的平均速率,得出:温度一定时,气体压强与分子密度成正比。从而理解绝对湿度的定义。

再通过实例和课本列举的数据,说明湿度的影响取决于空气中的水蒸气离饱和状态的远近,引入相对湿度的概念,对相对湿度的计算,学生易于接受,可让学生阅读教材及不同温度下饱和汽压的表格,做点练习题,理解这些知识。

\subsubsection{露点}

复习用降低温度使未饱和汽变为饱和汽的方
法及其道理,引入并讲清露点的概念:设温度为$t_1$时,空气中未饱和汽的密度为$\rho$; 降低它的温度至$t_2$, 若$t_2$温度下的饱和汽密度也是$\rho$, 在$t_2$温度下,原来的未饱和汽就变成饱和汽了.温度$t_2$就是空气的露点.

测定露点的实验,瓶面出现的凝结现象不易观察,温度计的可见度小,在课堂上演示时,可请学生参与操作和观察,把结果告诉大家。

由露点来求相对湿度,可以先讲一讲根据露点与原来气温的差值可以大致判断相对湿度是大还是小,再讲清根据露点计算相对湿度的方法。先要明确露点温度水的饱和汽压值,就是原来温度下水的未饱和汽压值,即原来温度下的绝对湿度,这可在不同温度下水的饱和汽压表中查出,再在这个表中查出原来温度下水的饱和汽压值,即可计算相对湿度。

\subsubsection{湿度计}

教师可以对干湿泡湿度计的使用作演示和示范讲解,再让学生阅读教材,了解它的构造、原理和优缺点。阅读前可以提出一些思考题,如为什么湿泡温度计的示数要低于干泡温度计的示数?使用时要读出哪些数据?怎样才能得出相对湿度?还可以给出一些简单的练习题,帮助学生了解怎样用湿度计测定相对湿度。

\section{实验指导}
\subsection{演示实验}
\subsubsection{研究饱和汽性质的实验}

实验装置如图5.4所示.$A$是一根竖直固定在木板上的
玻璃管,上下口用橡皮塞塞紧。$B$是一根上端有进液口$D$和进液阀门$E$的长直细玻管,直穿上下两橡皮塞的正中央,下端口通过橡皮管与长颈漏斗$C$相连。长颈漏斗$C$夹持在木板上,可以上下移动,木板上画有均匀刻度,圆筒上端橡皮塞装有一进水口$F$和插入一温度计$G$. 圆筒下端橡皮塞上装有一带阀门的出水管。以上装置可用J2257型气体定律演示器代替。
\begin{figure}[htp]
    \centering
      \includegraphics[scale=.7]{fig/5-4.png}
    \caption{}
\end{figure}

\paragraph{饱和汽压强的测量}
启开阀门$E$, 从长颈漏斗$C$的上端灌入清洁水银,提起漏斗,让水银排出$B$管内空气,使水银面上升至$B$管上端的阀门$E$, 恰有少量水银过阀门后,关闭阀门$E$. 在进液口$D$装入适量的乙醚(或其他被测物质),降低长颈漏斗,使$B$管内出现一段真空,两管水银面的高度差应为当地的大气压强$p_0$. 缓慢启开进液阀门$E$, 滴入$B$管适量的乙醚,再关上阀门$B$. 乙醚在$B$管水银面上方空间中蒸发;至$B$管水银面上剩有少许乙醚液为止,这时管内空间充满乙醚的饱和汽,读出此时$B$、$C$两管水银面的高度差$h$。$p=p_0\pm \rho gh$即为乙醚的饱和汽压值(当$B$管的水银面高于$C$管的水银面时,取“$-$”号,反之,取“$+$”号)。

\paragraph{饱和汽压不随体积变化}
保持室温不变,将长颈漏斗缓慢提升,可以看到饱和汽所在空间的体积变小,$B$管中水银面上乙醚液增加,而$B$、$C$管水银面高度差不变,这说明在温度一定时,饱和汽压与体积变化无关。

\paragraph{饱和汽压跟温度变化的关系}
把长颈漏斗置于适当位置,读出温度$t_1$和$B$、$C$水银面的高度差$h_1$, 这时的饱和汽压$p_1=p_0\pm\rho gh_1$. 从进水口$F$向圆筒中注满热水,在温度升高的过程中,$B$管水银面上方乙醚液将蒸发,若蒸发完了,则可启开进液阀门$E$, 继续滴入乙醚,待蒸发到水银面上留有少许乙醚为止,读出这时的温度$t_2$及$B$、$C$管水银面的高温差$h_2$, 饱和汽压为$p_2=p_0\pm \rho gh_2$, 得出饱和汽压随温度的升高而增大。

\subsubsection{温水在低压下沸腾}
如图5.5所示,在广口瓶中装入大半瓶温度约$95^{\circ}{\rm C}$左右的水,瓶口塞紧装有开口弯玻管的橡皮塞。玻管上口用橡皮管与注射器连接。用注射器抽气,使瓶内温水上方的气压降低,可看到温水沸腾现象。
\begin{figure}[htp]\centering
    \begin{minipage}[t]{0.48\textwidth}
    \centering
\includegraphics[scale=.7]{fig/5-5.png}
    \caption{}
    \end{minipage}
    \begin{minipage}[t]{0.48\textwidth}
    \centering
\includegraphics[scale=.7]{fig/5-6.png}
    \caption{}
    \end{minipage}
    \end{figure}

如图5.6所示,在烧瓶塞中插一个温度计$C$, 一个三
通管和一个直角弯管,三通管的一个支管带有阀门$A$的,另一支管用橡管接一大容量的注射器$B$; 直角弯管经橡皮管与
U型液体(水银或水)气压计$D$连接。

演示时,将阀门$A$启开与大气相通,用酒精灯将烧瓶内的水加热至沸腾,撤去酒精灯,待水中气泡消失、水温降至沸点以下时,关闭阀门$A$. 用注射器缓慢地抽气,可以看到U型气压计$D$左管液面上升,说明烧瓶内液面上方的压强减小;到一定时候,水重新沸腾。可以读出这时U型气压计两管液面的高度差$h$及温度计的示数$t$, 另测出当时的大气压强$p_0$, 由$p_0$和$h$可算出沸腾时对应的压强值$p$. 这样,可以定量研究$p$、$t$间的关系。

\subsection{学生实验}
\subsubsection{测定水的熔解热}

由于初中的学习,对这个实验已有一定基础。在明确实验目的、原理、操作步骤的前提下,还要注意以下几点。
\begin{enumerate}
\item 由于记录和计算涉及的物理量较多,学生要作充分的准备,分清哪些是实验中测得的?哪些是用表格查得的?哪些是计算得出的?便于正确处理数据,避免错记、漏重记。
\item $0^{\circ}{\rm C}$的冰块要在已准备好的冰水混合物中取出,尽量少带水分,迅速投入量热筒。
\item 温水的温度和质量、冰块的质量等要参照课本给出的参考数据,不要偏离过大,影响实验效果。
\item 搅动小筒中的水时,要用力适度,估计到温度计在水中的位置,以免损坏器材。实验时,温度计的示数从$t$. 下降到最低温度后,又缓慢回升,最低温度$t$就是热平衡时的温度。所以要对温度计密切注视,连续观察记录温度示数,才能正确确定平衡时的温度。
\end{enumerate}

\subsubsection{测定空气的相对湿度}

要明确这里使用的实验装置是一个简易的露点湿度计,通过测出空气的露点,计算相对湿度,实验的研究对象是金属盒外的空气。盒内的冰在水中熔解吸热,起降低温度的作用。

实验装置中光滑的环形金属片是为了易于观察金属
盒上的露滴,与之对比而设置的。要保证它的表面光亮清洁,跟金属盒绝热。

如果冰块的备取有困难,可以准备适量的铵盐(或尿素)代替,并准备一些浓硫酸。

投入水中的碎冰(不要太大块)或铵盐要适量.过少时,温度尚未降到露点,而溶质已熔解完毕,实验不能成功;过多时,温度降到露点以后的回升时间太长,搅拌的快慢,应视温度变化的具体情况而定。若用铵盐代替做实验时,发现温度回升太慢,可以滴入几滴浓硫酸,使温度回升加快。

为及时观察露滴的出现或消失,可用手指(或棉纱)在金属盒的同一地方来回擦动,将擦过的地方与环形金属片表面对比,以期观察是否有露滴出现或消失。一经发现,及时记下温度的示数。

\subsection{课外实验活动}
\subsubsection{测定水的汽化热}

实验误差的主要来源是设计原理的不够完善。这个实验把铝锅里的水每秒钟吸收的热量看作是不变的;实际上,水在升温和汽化(沸腾)的过程中,单位时间内吸收的热量是不同的。因为热交换物体间的温度差不同,单位时间传递的热量也不同,温差越大,传递的热量越多,温差越小,传递的热量越少,另外,随着水温的升高,水跟周围空气的温差增大,在单位时间内放出的热量也增加了。由于这两方面的原因,随着水温的升高,水在单位时间内获得的净热量就会减少,还有,水在
温度没有达到沸点之前就不断蒸发,达到沸点时水的质量已经比原来的少了,这些因素在本实验中都没有考虑,再则,水的温度在未升到沸点前已开始汽化,即水的质量随着温度的升高不断减少,实验误差的另一个来源是由测量的水的质量、水的初温和沸点,以及加热水至沸腾和全部汽化的时间,这是测量都可能产生误差。

\subsubsection{估计水升高的温度}

估计水升高的温度为摄氏几度。

\begin{enumerate}
    \item 测火柴杆的质量,把火柴杆看成是正四方柱体,测出火柴杆正方形横截面的边长$a$, 测出它的长度$b$, 则火柴的体积$V=a^2b$. 查出一般木材的密度$\rho$的值为$(0.4\sim 0.9)\x10^3{\rm kg/m^3}$, 火柴杆的质量$m=\rho a^2b$.
    \item 计算火柴杆燃烧放出的热量,可以认为,火柴杆完全燃烧放出的热量$Q_{\text{放}}=qm$. $q$是木材的燃烧值$1.26\x10^7{\rm J/kg}$。
    \item 估计热传递的效率.考虑到玻璃是热的不良导体,火柴杆燃烧时间内向周围空间传递的热量较多,估计$\eta=20\%$.
    \item 计算水的温度的升高$\Delta t$. 由于
    \[Q_{\text{吸}}=m_{\text{水}}C_{\text{水}}\Delta t,\qquad Q_{\text{放}}=q\rho a^2b\]
    由热平衡方程 $Q_{\text{吸}}=\eta Q_{\text{放}}$, 得
    \[\Delta t=\frac{\eta q\rho a^2b}{m_{\text{水}}C_{\text{水}}}\]
    代入数据,即可算得$\Delta t$.
\end{enumerate}



































































































