\chapter{物态变化}
\section{教学要求}
本章讲解的知识,是对初中学过的物态变化知识的扩展和加深,与生产、科研和日常生活实际有着密切的联系。

教材对熔解、凝固、汽化等现象,和熔解热、汽化热等概念,从分子运动论的观点和能量的观点作了定性的解释,是为了使学生对现象和概念理解得深刻一些;这也有利于培养学生的抽象思维能力。这一章的现象多、概念多,讲述时除了加强实验、增加感性知识外,还要多举生活中的实例,帮助学生理解。还要注意弄清楚某些具有共同特点的概念间的区别(如熔解热与汽化热)。

本章的教学可分为三个单元:第一单元,包括第一、二节,讲熔解和凝固的知识,第二单元,包括第三节到第七节,讲解蒸发、沸腾及饱和汽和气体的液化的知识,第三单元,包括第八、九节,讲解空气的湿度和露点。

下面对这一章的教学内容作一些具体说明。

晶体和非晶体在熔解时的区别初中已学过,只要求作简单的复习,课本对此从分子结构不同所作的解释,以及对熔
解热用分子运动论和能量转化的知识所作的说明,可以使学生从理论上加深对这些现象的认识,同时使他们了解理论的应用,从而巩固理论知识。

饱和汽的知识是理解液体的沸腾、气体的液化、空气的湿度、露点等许多现象的基础,需要重点讲解,它又是学生新接触的理论性较强的内容,也是教学的难点,理解饱和汽的知识,关键在于讲好动态平衡的概念。

液体沸腾的条件和沸点跟压强的关系,课本是通过实验观察得出结论的。至于为什么当饱和汽压等于外界压强时液体才沸腾,这个问题在中学阶段很难讲清楚,教材中也没有进一步分析这个问题,希望在教学中注意掌握,不做过高的要求。

课本对汽化热也用分子运动论和能量转化的知识作了说明,液体汽化时用于克服外界压强所做的功不能忽略,沸点变化时汽化热差别较大,这些与熔解热的不同点,要向学生指明。

空气的湿度和露点,在生产和生活实际中常常用到,课本讲解这些知识,就是为了给学生理解一些实际现象打下一个初步基础。

根据以上分析,本章的教学要求是:
\begin{enumerate}
\item 了解晶体和非晶体熔解和凝固时的不同,掌握熔解热的概念。
\item 知道蒸发和沸腾的区别,了解沸腾的条件及沸点跟压强的关系,掌握汽化热的概念。
\item 知道什么是饱和汽,理解动态平衡的概念.了解饱和汽的压强(或密度)跟温度有关,跟体积无关,知道气体液化的方法,了解临界温度的概念。
\item 理解绝对湿度、相对湿度和露点等概念.
\end{enumerate}

























