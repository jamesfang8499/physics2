
\chapter{气体的性质}\minitoc[n]
\section{教学要求}
从实验出发,建立气体实验定律,对实验定律进行研究而获得理想气体的状态方程,最后从微观上利用气体分子运动论和统计方法解释气体实验定律,本章教材的这种安排,较鲜明地体现了实验的作用和理论的指导意义,体现了宏观和微观的统一,这是本章教材的特点,注意到这一点,无疑对学生有科学研究方法上的启迪.

本章可分为四个单元,第一单元由第一节至第四节,从介绍气体的三个状态参量开始,研究气体的两个实验定律并引出热力学温标,第二单元包括五、六两节,主要研究理想气体的状态方程和克拉珀龙方程,第三单元由七、八两节组成,介绍气体分子运动的特点并对气体的实验定律进行定性的微观解释,第四单元为九、十两节,从能量的角度研究理想气体并讨论了理想气体在几个等值过程中内能的变化.

气体的三个状态参量,体积、压强、温度是研究气体性质的基础,在教学中,要求学生理解这三个状态参量的意义,由于压强的计算稍难一点,因此在教学中要多花一些力气搞清
压强的计算和单位间的关系.

气体三个实验定律反映了气体在温度不太低、压强不太高的条件下状态变化所遵循的规律,是建立理想气体状态方程的依据,因此,要求学生很好的掌握,对气体等温变化的图线和气体等容变化的图线的物理意义要清楚地理解.

教材是利用气体等容变化的$p$-$t$图线交于${\rm -273^{\circ}C}$引出绝对零度从而建立热力学温标的,热力学温标的建立对简化查理定律并导出理想气体状态方程起了重要作用 因此,要使学生掌握热力学温度和摄氏温度的换算关系,理解热力学温标和绝对零度的意义.

理想气体的状态方程反映了一定质量的理想气体三个状态参量间的变化关系,是本章的重点知识,对于进一步学习和生产实践都有重要的意义,要求学生巩固掌握,克拉珀龙方程是本章的重要知识,但不是重点,要求学生掌握,但不要去做过多过深的习题,以免加重学生的负担.

关于气体分子运动的特点,以及对气体实验定律的微观解释,教材在介绍知识的同时,渗透一些统计的观点,由于这部分内容比较抽象,在中学也不可能讲解透彻,因此,不宜讲得过多过深,不宜要求过高.

理想气体的内能一节介绍了理想气体的微观模型,指出了它的内能只跟温度有关,跟气体体积无关,了解了理想气体的微观模型,就能理解为什么实际气体在温度不太低,压强不太高时,其性质跟理想气体接近,并为进一步从能量角度研究理想气体打下基础.关于理想气体内能的变化,应根据学生情况决定讲与不讲,讲到什么程度.这一节内容,对热力学第
一定律用于理想气体的几个等值过程做了粗略的介绍,利于
扩大学生的知识面,但限于学生实际水平,只能要求他们定性地了解这些知识.

这一章的教学要求是:
\begin{enumerate}
 \item 理解描述气体状态的三个参量(体积、压强、温度)的意义,会计算气体的压强.
  \item 掌握玻意耳-马略特定律、查理定律、盖·吕萨克定律,理解气体等温变化的p-V图象和气体等容变化的p-t、p-T图象的意义,会用这三个定律计算有关的问题.
  \item 理解热力学温标和绝对零度的意义,掌握热力学温度和摄氏温度的换算关系.
  \item 巩固掌握理想气体的状态方程,掌握克拉珀龙方程,会用它们来计算有关问题.
  \item 了解气体分子运动的特点,了解气体压强的微观解释,会用气体分子运动论对气体的三个实验定律进行微观解释.
  \item 了解理想气体的微观模型,了解理想气体的内能只跟温度有关,跟体积无关.了解理想气体在几种等值过程中内能的变化情况.
\end{enumerate}
 
\section{教学建议}
\subsection{第一单元}
\subsubsection{气体的状态和状态参量}

本节研究了描述气体状态的三个参量,提出了本章研究的中心课题,即确定三个状态参
量之间的变化规律,指出了研究方法:保持其中一个量不变研究其他两个量的关系,然后再确定三个状态参量之间的关系.让学生在学习本节内容时,明确以上各点能够增强学生学习的目的性.

在介绍描述气体状态的状态参量时,应使学生明确地认识到,在物理学中因研究的问题不同,所用的物理量也不同.在热学中研究气体的性质,所用参量有体积、压强和温度.

在讲解本节内容时,应注意三点:
\begin{enumerate}
\item 气体体积是指气体分子充满的空间,即容器的容积.这个容积不是气体分子本身的体积之和,气体分子之间是有空隙的.
\item 对气体压强的计算方法,单位换算要给予足够的重视.因为它是后面学习气体定律和气态方程的基础,正确地确定气体的压强往往是应用气态方程解决问题的关键,恰当地选择压强的单位可以避免繁杂的计算.在教学中要多花些力气,学生也应做一定练习,务求掌握气体压强的计算方法和单位之间的换算关系.
\item 教材中提到“对一定质量的气体来说,如果体积、压强和温度这三个量都不改变,我们就说气体处于一定的状态中”.这个“状态”指的是气体的热动平衡态,有关热动平衡态,在课堂上无需涉及,在这里,只要求学生理解上面那段话的意思就行了.
\end{enumerate}

\subsubsection{气体的等温变化}
 玻意耳-马略特定律 节利用实验研究一定质量气体的等温变化,总结出玻-马定律,对等温线作了介绍,最后指明了玻-马定律的物理意义和适用
 范围.

 讲好玻-马定律的关键是做好演示实验,用教材中介绍的装置,也可以验证在温度不变的条件下,$pV=$恒量.

 在做演示实验前,要引导学生仔细观察实验装置,明确指出研究对象是$A$管中密闭的气体,提出下列问题让学生思考回答:
 \begin{enumerate}
 \item 如教材39页图3.5所示,$A$管中气体的压强的表达式是怎样的?
 \item $A$管中的气体体积怎样确定?
 \item 为了保证实验中气体恒温,$B$管的升降快慢有什么要求?为什么?实验中的恒温是什么温度?
 \end{enumerate}

 该实验不一定采用书上介绍的装置,也可用其他仪器进行,不论采用哪种办法,其共同点是保证气体的定质量和等温变化的条件. 

 对于等温线,主要讲清它表明了一定质量气体在等温条件下压强和体积的变化关系,不要求利用图线进行定量计算.教材中给出了不同温度下的几条等温线,主要是说明$pV=$恒量这个公式中的恒量跟温度的关系,得出对于一定质量的气体,“恒量随温度的增高而增大”的结论.显然这个恒量不是一个普适恒量.因此,要得出$p$、$V$、$T$三个状态参量之间的关系,还必须研究$P$、$T$或$V$、$T$之间的规律.关于这一点,教师可对比牛顿第二定律的研究过程:在$F$、$m$、$a$三个物理量中先研究加速度和力的关系,再研究加速度和质量的关系,从而确立了三个量之间的关系,引导学生注意这种控制条件的研究方法.

 教材选用了一个漏气的例题,这是一个有利于培养学生分析能力的题目,应该引导学生分析讨论,由于该题是第一
 个利用玻-马定律求解的题目,教师一定要明确指出,确定研究的对象是哪一定量的气体,对它进行变化前后的状态分析是正确解题的前提,教材也为此做了很好的示范,教师要充分注意这个问题,使学生一开始就养成良好的习惯、掌握正确的分析方法,由于本章习题涉及的物理量多,状态往往在两个或两个以上,做到这一点,尤为重要.

\subsubsection{气体的等容变化~~查理定律}

本节利用实验定性地归纳出在体积不变的条件下,一定质量的气体的压强随温度的升高而增大,随温度的降低而减小,然后给查理定律的表述和公式,给出气体等容变化的图线,最后明确查理定律的适用范围.

 教学中应对查理定律演示实验进行精心准备,使实验达到定量的效果,用实验所取得的数据作出如教材第47页的$p$-$t$图线,其延长线在实验误差范围内交于$-273^{\circ}{\rm C}$. 这样做的好处是:
 \begin{enumerate}
     \item 能突出物理规律的实验研究方法;
     \item 由实验图线获得$-273^{\circ}{\rm C}$的外推点,为下节研究热力学温标打下基础;
     \item 查理定律可以由图象总结出来.
 \end{enumerate}

 由于学生已学过斜截式的直线方程,比较容易总结出查理定律的教学表达式:由图线写出方程
\[p=kt+p_0\]
其中$p_0$为$0^{\circ}{\rm C}$时气体的压强,斜率$K=\frac{p_0}{273}$,
所以
\[p=\frac{p_0}{273}\cdot t+p_0\]
即
\[p=p_0\left(1+\frac{t}{273}\right)\]

为了能定量地进行实验,一要用气压表测定实验时的大气压;二要在$0^{\circ}{\rm C}$以下得出一组$p$、$t$数据,才能使$p$-$t$图象较准确地交$t$轴于${\rm -273^{\circ}C}$.为此,可用食盐加碎冰溶化获得零下20多摄氏度的低温;三要注意给出充分地时间让气体跟周围环境达到热平衡.

同讲解玻-马定律一样,教材在讲解查理定律后又强调了定律的适用范围是压强不太大,温度不太低,这点仍须向学生明确指明,物理定律都有它的适用范围.超出了这个范围,物理现象所遵从的规律将发生变化,不再是原来的规律了.对于气体定律的适用范围,学过理想气体的微观模型以后,就可以进一步理解它的道理了.

\subsubsection{热力学温标}

由查理定律外推而引入新的温标,绝不意味着查理定律的适用范围可以“外推”,在低温下,查理定律不适用了,这点要再次强调.

“外推点”定为热力学温标的零度,也绝不是一次实验的结果,而是对不同气体做多次实验所作出的图线都交$t$轴于${\rm -273^{\circ}C}$, 这才赋予它以新的物理意义,引出绝对零度,这一点在教学中要交代清楚.

在引出热力学温标后,要强调“就每一度的大小来说,热力学温度和摄氏温度是相同的”.这就是说,虽然热力学温度跟摄氏温度间的关系为$T=t+273.15$(K), 但就\textbf{同一温度变化}来看,利用热力学温标与利用摄氏温标所表示的数值是相同的,即$\Delta T=\Delta t$. 这一点在计算题中经常用到.

由实验总结出查理定律,又由查理定律“外推”到零压强而引入热力学温标、绝对零度,这一方面体现了实验的重要
性,另方面也体现了理论的指导作用,正是由于这种理论上的外推使人们获得了绝对零度的概念,导致了人类不断向绝对零度进军的历史进程,向学生介绍这点可以启迪思维,加深对物理研究方法的认识,结合讲解一些人类向绝对零度进军的情况,可以大大提高学生兴趣,开阔学生眼界.

\subsection{第二单元}
\subsubsection{理想气体的状态方程}

本节利用玻-马定律和查理定律推出一定质量气体的状态方程,提出了理想气体这一概念,说明了状态方程包含了玻-马定律和查理定律,并由它推出了另一个实验定律——盖·吕萨克定律.

对盖·吕萨克定律,教材中叙述得比较简单,教师可以补做演示,使学生的印象深刻些.

要再次强调状态方程的适用范围,强调理想气体是一种科学的抽象,一个理想的物理模型,要使学生明确地认识到,理想气体虽然并不存在,但它是实际气体在一定程度上的近似,在压强不太大,温度不太高的条件下可把实际气体作为理想气体来处理.

课后布置教材习题4, 根据玻-马定律和盖·吕萨克定律推出一定质量的理想气体的状态方程$pV/T=$恒量,向学生指明,只要初末状态确定,不论经过哪两个等值过程,都可以推出状态方程.这种练习,对于培养学生发散性思维是有帮助的.

\subsubsection{克拉珀龙方程}

克拉珀龙方程是任意质量的理想气体的状态方程,推导克拉珀龙方程时,要注意启发性,注意启
发的层次.教材是按下列层次安排的:
\begin{enumerate}
\item 从$pV/T=$恒量开始讨论,说明这个恒量与气体的质量和气体的种类有关;
\item 研究1摩尔任何气体的$p_0V_0/T_0$值,即摩尔气体恒量$R$, 从而得到1摩尔理想气体的状态方程$pV=RT$;
\item 由标准状态下,$n$摩尔气体占有体积为$nV_0$, 利用摩尔气体恒量$R$导出任意质量的理想气体的状态方程,即克拉珀龙方程$pV=\frac{m}{M}RT$.
\item 最后说明气体满足方程的条件.
\end{enumerate}

教材这样安排,层次是清楚的,启发性也较强.在讲解中,要指出摩尔气体恒量是热学中的一个重要常数,限于学生知识水平,其重要性不可能在课堂上分析,但值得注意的是,在讲解中不要把$R=p_0V_0/T_0$误认为是$R$的物理意义,上式仅仅是$R$的计算方法,由于在标准状态下,1摩尔的任何气体体积$V_0$都是22.4升,上式表明$R$是一个普适恒量,$R$的物理意义是:1摩尔理想气体在压强不变时,温度升高1开对外界所做的功.R的物理意义在讲授中不宜提出,更不能给出错误的解释,这一点应该注意.

在应用克拉珀龙方程解题前,还必须指出,由于$R=8.31$焦/摩·开$=0.082$标准大气压·升/摩·开,因此,$p$、$V$的单位必须与选用的$R$的单位相适应.

利用三个实验定律和两个状态方程解题时,要注意训练学生对气体变化前后的状态进行分析,教师要做出表率.

\subsection{第三单元}
本单元通过对气体分子运动的特点的分析,对气体实验定律进行微观解释,由于本单元的内容较为抽象,因此,一方面要严格控制,不要增加教材以外的内容;另一方面要使学生对分子间距离的大小和分子间的频繁碰撞在头脑中有深刻的、形象的认识.

讲解气体分子运动的特点要渗透一些统计的观点,要使学生知道,个别分子的运动具有偶然性,但大量分子在总体上都服从确定的规律,这种规律是统计性的,叫做统计规律.

为了讲好“气体实验定律的微观解释”,建议在讲课开始时,复习压强的概念、公式及弹性小球对墙壁的碰撞给与墙壁的冲量的计算,为正课的讲授扫清一些障碍.

压强的微观解释是本节的重点,在讲解这个问题时,要使学生体会到,虽然个别分子的运动服从力学规律,个别分子碰撞器壁时所产生的冲力也要用力学规律来计算,但涉及大量分子的集体行为却要用到统计观点,因此,不是一种纯粹的力学问题.

另外,应注意形成气体压强的微观图景,教材是以雨滴对伞的压强为例来说明的,为了加深理解,建议课堂上进行模拟气体压强的演示实验,以增加学生的实感.

\subsection{第四单元}
本单元的必学教材为第九节,该节主要建立理想气体的
微观模型,并指出理想气体的内能就是气体所有分子热运动的动能的总合,不存在分子势能,因此理想气体的内能仅与温度有关而与体积无关,建议在讲解本节内容前,认真复习第二章第一节“物体的内能”,让学生搞清分子动能、分子势能是由什么因素决定的,在此基础上,本节内容是容易理解的.

第十节是选学教材,对于基础较好的班可以选用,本节研究各种等值过程中理想气体的内能的变化,其研究的手段是利用热力学第一定律,该节的全部内容可列为下表.

\begin{table}[htp]
\caption{理想气体等值过程的讨论}
\begin{tabular}{cp{.32\textwidth}p{.18\textwidth}c}
\hline
过程名称& 各等值过程的物理现象 & 热力学第一定律的形式&$p$-$V$图\\
    \hline
等温过程&内能不变$\Delta E=0$,\par 吸热则体积膨胀,对外做功;放热则体积缩小,外界对气体做功
&$W+Q=0$&\raisebox{-.65\height}{\tikz[>=latex, yshift=-3cm]
 {\draw[<->](0,2.5)node[right]{$p$}--(0,0)--(2.5,0)node[above]{$V$};\draw(.34,2) [bend left=-35] to node[right]{等温线} (2,.34);
 }        }
 \\
\hline
等容过程&气体不做功$W=0$,\par 吸热则内能增加;放热则内能减少
&$Q=\Delta E$&\raisebox{-.45\height}{\tikz[>=latex]
 {\draw[<->](0,2.5)node[right]{$p$}--(0,0)--(2.5,0)node[above]{$V$};\draw(1,0) to node[right]{等容线} (1,2.5);
 }        }
\\
\hline
等压过程&膨胀时,气体对外做功,其密度减小,为保持压强不变,必须吸热升温,故内能增加;压缩时,情况相反
&$W+Q=\Delta E$&\raisebox{-.85\height}{\tikz[>=latex]
 {\draw[<->](0,2.5)node[right]{$p$}--(0,0)--(2.5,0)node[above]{$V$};\draw(0,1) to node[above]{等压线} (2.5,1);
 }        }
\\
\hline
绝热过程&气体与外界无热交换,$Q=0$.
体积膨胀,对外做功,内能减少;体积压缩,外界对气体做功,内能增加
&$W=\Delta E$&\raisebox{-.85\height}{\tikz[>=latex]
 {\draw[<->](0,2.5)node[right]{$p$}--(0,0)--(2.5,0)node[above]{$V$};\draw(.34,2) [bend left=-35] to node[right]{绝热线} (2,.34);
 }        }
\\
\hline
\end{tabular}
\end{table}

\section{实验指导}
\subsection{演示实验}
\subsubsection{气体压强的模拟实验}

仪器可选用“气体分子运动模拟实验器”. 该仪器如图3.1所示.
\begin{figure}[htp]\centering
    \begin{minipage}[t]{0.3\textwidth}
    \centering
    \includegraphics[scale=.8]{fig/3-1.png}
    \caption{}
    \end{minipage}
    \begin{minipage}[t]{0.58\textwidth}
    \centering
    \includegraphics[scale=.8]{fig/3-2.png}
    \caption{}
    \end{minipage}
    \end{figure}

其中的钢球模拟气体分子.在玻璃筒内放入50—80个钢球,接通电源,选择适中的电压,由于活塞的振动引起钢球的无规则运动,不断地冲击浮动板使之升起到一定的高度,这就定性地模拟了大量气体分子频繁地碰撞器壁产生持续的均匀的压力,形成对器壁的压强,如图3.2所示.

\begin{figure}[htp]
    \centering
\includegraphics[scale=.8]{fig/3-3.png}
    \caption{}
\end{figure}

如果没有上述仪器,可用托盘天平来演示,如图3.3所
示,将托盘天平中的一盘反扣在支架上,调平后,在正放的托盘里放入适当质量的砝码,用小钢球(如自行车轴承里的钢球,或用豌豆)自反扣托盘的上方徐徐释放,不断地冲击托盘,使天平达到平衡,这就定性地说明,大量气体分子对器壁的频繁碰撞,产生持续的,均匀的压力,形成对器壁的压强.

\subsubsection{温度不变时气体的压强跟体积的关系}

做教材图3.5的实验时,应注意:
\begin{enumerate}
\item 灌注水银时,要利用小漏斗和细铁丝将水银慢慢注入,尽量避免水银中带有空气泡;
\item $A$、$B$管与橡皮管连接处要用细铁丝绑紧,防止由于装入水银而脱落;$A$管的阀门$a$要涂上凡士林,以保证良好的密闭性;
\item 实验中,$A$、$B$管始终保持竖直,$B$管的升降移动要缓慢,保证有足够的时间使$A$管中的气体跟周围空气进行热交换,以便保持等温条件.
\item 水银不浸润玻璃,液面成凸弯月面,读数要读水银凸面顶端的刻度数.
\end{enumerate}

\subsubsection{体积不变时气体的压强跟温度的关系}

做教材图3.9的实验时,其注意事项与玻-马定律注意事项类似,不再赘述,建议在$B$管上套一个橡皮圈,标记水银面的位置,即密闭气体的定体积标志,每次改变容器的水温,让其有足够的时间使烧瓶内气体与水达到热平衡,再调整$A$管的高度,使$B$管中水银面恰好在橡皮圈标志处,这样做可以使学生更清楚地观察到实验在等容条件下进行.

\subsubsection{压强不变时气体的体积跟温度的关系}

定性演示可利用气体在大气压下热胀冷缩实验进行,如图3.4所示的简单仪器即可做这个实验,用双手握住烧瓶,瓶内气体膨胀,水滴右移;用冷湿毛巾包住烧瓶,瓶内气体收缩,水滴左移.为增加可见度,水滴应染色.
\begin{figure}[htp]
    \centering
\includegraphics[scale=.8]{fig/3-4.png}
    \caption{}
\end{figure}

定量演示可利用针筒,其装置同本章的学生实验.针筒竖放固定,内筒上不加任何压载物,当筒内密闭了一定质量的气体后,若忽略内筒所受的重力,则气体的压强恒为当地的大气压值,这就保证了压强不变的条件.将针筒逐次放入温度不同的水中,每次都要保持针筒竖放、水淹没筒内气体并保证必要的热平衡所需的时间,就可以从温度计和针筒上的刻度读出各次实验的$t$、$V$值,将$t$值换为$T$值,可证明在实验误差范围内
\[\frac{V_1}{T_1}=\frac{V_2}{T_2}=\frac{V_3}{T_3}=\cdots\]

上式即盖·吕萨克定律.

\subsection{学生实验}
本章有两个学生实验:“验证玻意耳-马略特定律”和“验证气体状态方程”. 教材上选用针简作为这两个学生实验的主要仪器,优点是在学生实验中取掉了水银,用针筒完全可以较准确地做好这两个实验,下面提出一些注意事项供参考.

\begin{enumerate}
    \item 针筒一般选用50毫升规格为宜,针筒的内外筒的配合应该是“密而不紧”.把注射器小孔打开,竖直拉起内筒,针筒能够在自重的作用下慢慢下落,这就达到“不紧”的要求.将内筒拉起一定高度后,密封针筒小孔,这时再用力将内筒拉出或压入一段距离,放手后,内筒能恢复到原来的位置,这就说明针筒的密闭性是较好的,达到了“密”的要求,不能满足上述条件的针筒是不能选用的.
\item 消除内、外筒间的梗阻,做上项选择实验或正式实验前,都务须用细软的、不留毛屑的布将内外筒擦净,以免尘粒毛屑之类的小物梗阻内筒的运动、
\item 提高针筒的密闭性是两个实验成败的关键,为此,可以在擦净内外筒后,在其间注进少许缝纫机油或纯净的轻机油,千万不要用水代替机油,水蒸气会破坏针筒内气体的定质量条件.
\end{enumerate}

\section{习题解答}

\subsection{练习一}
\begin{enumerate}
	\item 什么叫气体的压强?举出气体对器壁有压力作用的几个实例.

\begin{solution}
    气体作用在器壁单位面积上的压力叫做气体的压强.充氢气的气球,氢气对气球内壁有压力作用;内燃机气缸中的燃气对活塞和气缸壁有压力作用,正是这个压力,推动活塞做功;高压锅中的蒸汽,对锅内壁有压力作用,这个压力随着蒸汽的增多而增大,到一定限度,推开高压阀,使高压蒸汽喷出.
\end{solution}
	\item 大气压为750毫米汞柱时,等于多少帕?

\begin{solution}
\[750{\rm mmHg}=750\x \frac{1.013\x 10^5}{760}=1.00\x 10^6{\rm Pa}\]
\end{solution}
		\item 在教材图3.2中,水银柱的长度为19厘米,大气压为
	760毫米汞柱,玻璃管开口向上竖直放置时,被封闭的气体的压强等于多少毫米汞柱?开口向下竖直放置时,等于多少毫米汞柱?


\begin{solution}
玻璃管开口向上竖直放置时,气体压强为
\[p_1=p_0+p_k=760+190=950{\rm mmHg}\]
玻璃管开口向下竖直放置时,气体压强为
\[p_2=p_0-p_k=760-190=570{\rm mmHg}\]
\end{solution}

	\item 图3.5是测量气体压强的水银压强计,两端开口的U形管内装入水银,$A$管跟容器连接.已知大气压$p_0$和两管中水银面的高度差,就可以知道容器中气体的压强.大气压为$1.013\times 10^5{\rm Pa}$,图甲和图乙中的$h$都是10厘米,分别求出这两种情形中气体的压强是多少帕.
    \begin{figure}[htp]\centering
        \includegraphics[scale=.8]{fig/3-5.png}
        \caption{水银压强计}
    \end{figure}	

\begin{solution}
    依题意,甲图中容器的压强应该等于大气压强和$h$高水银柱产生的压强之和,乙图中容器的压强为上两项之差.即:
\[\begin{split}
    p_{\text{甲}}=p_0+\rho gh
    &=(1.013\x10^5+13.6\x10^3\x9.8\x0.10)\\
    &=1.14\x 10^5{\rm Pa}\\
    p_{\text{乙}}=p_0-\rho gh
    &=(1.013\x10^5-13.6\x10^3\x9.8\x0.10)\\
    &=0.88\x 10^5{\rm Pa}\\
\end{split}\]    
\end{solution}
	\item 在图3.4所示的几种情形中,被封闭的气体$A$的压强分别是多少帕?大气压为$1.013\times 10^5{\rm Pa}$.
\begin{figure}[htp]\centering
	\includegraphics[scale=.8]{fig/3-6.png}
	\caption{}
\end{figure}	

\begin{solution}
\[\begin{split}
    p_{1}=p_0-\rho gh_1
    &=(1.013\x10^5-13.6\x10^3\x9.8\x0.20)\\
    &=7.4\x 10^4{\rm Pa}\\
    p_{2}=p_0+\rho_{\text{水}} gh_2
    &=(1.013\x10^5+1.0\x10^3\x9.8\x 20)\\
    &=2.97\x 10^5{\rm Pa}\\
    p_{3}=p_0+\rho_{\text{水}} gh_3
    &=[1.013\x10^5+1.0\x10^3\x9.8\x(0.85-0.10)]\\
    &=1.87\x 10^5{\rm Pa}\\
\end{split}\]  
\end{solution}
	\item 举出气体状态发生改变的几个实例.

\begin{solution}
 柴油机汽缸中的气体,在第二冲程时被急骤压缩,气体的三个状态量都发生变化;氢气球破裂,氢气溢出,其体积和压强发生明显变化,由于体积膨胀,温度也要发生变化.   
\end{solution}
\end{enumerate}

\subsection{练习二}
\begin{enumerate}
		\item 把打气筒的出口堵住,往下压打气筒的活塞,会感到越往下压越费劲,怎样解释这个现象?

\begin{solution}
打气筒的出口被堵住后,简内密闭了一定质量的空气,往下压打气筒的活塞,可将其近似地看成温度不发生改变(实际上温度有一点变化),越往下压活塞,气体的体积越小.根据玻意耳-马略特定律,这时气体的压强增大,所以越往下压感到越费劲.
\end{solution}
	\item 某个容器的容积是5升,里面所装气体的压强是10标准大气压,如果温度保持不变,把容器的开关打开以后,这些气体会有多大体积?容器里剩下的气体是原来的百分之几?设外界压强为1标准大气压.
	
    \begin{solution}
根据题意,$V_1=5$升,$p_1=10$标准大气压,当容器的开关打开以后,气体的体积为$V_2$,$p_2=1$标准大气压,根据玻意耳-马略特定律:$p_1V_1=p_2V_2$

$\therefore\quad V_2=\dfrac{p_1V_1}{p_2}=\dfrac{10\x 5}{1}=50{\rm L}$

容器内剩下的气体是$V_1=5$升,气体的总体积为$V_2=
50$升,故容器里剩下的气体是原来的
\[\frac{5}{50}=\frac{1}{10}=10\%\]       
    \end{solution}
	\item 在上题里,打开容器的开关以后,气体的密度怎样改变?设上题里容器里剩下的气体的密度是$\rho_2$,原来容器里气体的密度是$\rho_1$,那么,密度之比$\rho_2/\rho_1$是多大?
	
\begin{solution}
 由于原容器内的压强是10标准大气压,外界压强是1标准大气压,打开容器后,气体体积增大,因气体质量不变,故密度变小.
 \[\frac{\rho_2}{\rho_1}=\frac{m/V_2}{m/V_1}=\frac{V_1}{V_2}=\frac{5}{50}=\frac{1}{10}\]   
\end{solution}
	\item 在密闭圆筒的中央有一个活塞(图3.7),活塞两边
	封闭着两部分气体,它们的压强都是750毫米汞柱,现在用
	力把活塞向右移动,使活塞右边气体的体积为原来的一半,那么活塞两边气体的压强差是多大?假定气体的温度不变.
\begin{figure}[htp]\centering
	\begin{tikzpicture}[>=latex]
\draw [pattern=north east lines](2.4,0) rectangle (2.6,2);	
  \pgfsetlinewidth{4pt}
\pgfsetinnerlinewidth{3pt}
\draw  (0,0) rectangle (5,2);
\end{tikzpicture}
\caption{}
\end{figure}

\begin{solution}
设活塞移动前后左边的压强和体积分别为$p_1$、$p'_1$、$V_1$、$V'_1$, 活塞右边的压强和体积分别为$p_2$、$p'_2$、$V_2$、$V'_2$,其中$p_1=p_2=750$毫米汞柱,$V_1=V_2$,
\[V'_1=\frac{3}{2}V_1,\qquad V'_2=\frac{1}{2}V_2\]

根据玻意耳-马略特定律,对左边的气体有
    \[p_1V_1=p'_1V'_1\]
$\therefore\quad p'_1=\dfrac{V_1}{V'_1}p_1=\dfrac{V_1}{\frac{3}{2}V_1}p_1=\dfrac{2}{3}p_1=\dfrac{2}{3}\x 750=500{\rm mmHg}$
    
    对右边的气体有
    \[p_2V2=p'_2V'_2\]
    
$\therefore\quad p'_2=\dfrac{V_2}{V'_2}p_2=\dfrac{V_2}{\frac{1}{2}V_2}p_2=2p_2=2\x 750=1500{\rm mmHg}$
    
    活塞两边的压强差
\[\Delta p=p'_2-p'_1=1500-500=1000{\rm mmHg}\]
    
\end{solution}

	\item 在教材图3.2中,水银柱的长度为19厘米,大气压为760毫米汞柱,玻璃管是粗细均匀的,玻璃管开口向上竖直放置时,被封闭的气体柱长15厘米,当开口向竖直放置时,被封闭的气体柱的长度是多少?
	
\begin{solution}  
设玻璃管开口向上竖直放置时,气体的压强和体积分别为$p_1$、$V_1$, 玻璃管开口向下竖直放置时,气体的压强和体积分别是$p_2$、$V_2$, 玻璃管的横截面积为$S$.
    则:
\[\begin{split}
      p_1&=760+190=950{\rm mmHg}\\
    V_1&=\ell_1S=15{\rm cm}\x S\\
    p_2&=760-190=570{\rm mmHg}\\
    V_2&=\ell_2S  
\end{split}\]    
    
玻璃管在倒置过程中,气体的温度可以认为不变.根据玻意耳-马略特定律:
\[p_1V_1=p_2V_2\]
$\therefore\quad V_2=\dfrac{p_1}{p_2}V_1=\dfrac{p_1}{p_2}\x 15{\rm cm}\x S$

即:$\ell_2S=\dfrac{p_1}{p_2}\x 15{\rm cm}\x S$

$\therefore\quad \ell_2=\frac{950}{570}\x 15=25{\rm cm}$

当开口向竖直放置时,被封闭的气体柱的长度是25厘米.
\end{solution}  
	\item 在下端封闭的竖直玻璃管里有一段4厘米长的水银柱,水银柱下面封闭着6${\rm cm^3}$的空气,玻璃管的横截面积是0.1${\rm cm^2}$.如果再向管里装入27.2克水银,那么,封闭在水银
 柱下面的空气柱有多高?设大气压为760毫米汞柱.

\begin{solution}
当只有4厘米长的水银柱封闭玻璃管时,气体的压强和体积分别为$p_1$、$V_1$,则
\[\begin{split}
    p_1&=76+4=80{\rm cmHg}\\
    V_1&=6{\rm cm^3}
\end{split}\]

再向管里装入27.2克水银后,由于管的横截面积$S=0.1{\rm cm^2}$
,故27.2克水银在管内的高度
\[h'=\frac{m}{\rho S}=\frac{27.2}{13.6\x 0.1}=20{\rm cm}\]
若这时气体的压强和体积为$p_2$、$V_2$,
则$$p_2=76+4+20=100{\rm cmHg}$$
可认为气体的温度不变,由玻意耳-马略特定律:
\[p_1V_1=p_2V_2\]
$\therefore\quad 
V_2=\dfrac{p_1}{p_2}V_1=\dfrac{80}{100}\x 6=4.8{\rm cm^3}$

此时空气柱的长度为$H$,
\[H=\frac{V_2}{S}=\frac{4.8}{0.1}=48{\rm cm^3}\]
\end{solution}
	\item 一个足球的容积是2.5升.用打气筒给这个足球打气,每打一次就把1标准大气压的空气打进去125${\rm cm^3}$,如果足球在打气前内部没有空气,打了40次以后,足球内部空气的压强有多大?假定空气的温度不变.
	
\begin{solution}
将打入足球的那部分空气作为研究对象,打入足球前空气的压强和体积为$p_1$、$V_1$, 打入足球后空气的压强和体积为$p_2$、$V_2$.
则$p_1=1$标准大气压,$V_1=40\x125{\rm cm^3}=5{\rm L}$.$V_2=2.5$升.

由题设条件假定空气的温度不变,根据玻意耳-马略特定律
\[p_1V_1=p_2V_2\]
得:
\[p_2=\frac{V_1}{V_2}p_1=\frac{5\x 1}{2.5}\text{标准大气压}=2\text{标准大气压}\]
\end{solution}
\end{enumerate}

\subsection{练习三}

\begin{enumerate}
	\item 炎热的夏天,打足了气的自行车胎在日光曝晒下有时会胀破,解释这个现象.

\begin{solution}
  自行车胎在炎热的夏天被日光曝晒,车胎里的气体的温度上升.车胎的容积在打足了气后已不能再增大,根据查理定律,气体体积一定时,一定质量的气体的压强跟热力学温度成正比.所以气体的压强将增大.当压强达到车胎所能承受的最大压强时,温度再高车胎就会被胀破.
\end{solution}
\item 乒乓球挤瘪后,放在热水里泡一会,会重新鼓起来.解释这个现象.

\begin{solution}
    挤瘪的乒乓球在热水里浸泡时,乒乓球内的空气温度升高,在一个极短的时间内可以认为体积不变,根据查理定律,球内空气的压强增大,当球内压强到达一定值时,乒乓球就会鼓起一点,温度再升高,根据相同的道理,乒乓球会再鼓起一点……, 一会儿,乒乓球会重新鼓起来.
\end{solution}
\item 一定质量的氢气在0$^\circ$C时的压强是700毫来汞柱,它在30$^\circ$C时的压强是多大?压强为650毫米汞柱时它的温度
是多少摄氏度?保持氢的体积不变.

\begin{solution}
设氢气在$0^{\circ}{\rm C}$时的压强和热力学温度分别为$p_1$、$T_1$, 在$30^{\circ}{\rm C}$时的压强和体积分别为$p_2$、$T_2$, 压强650毫米汞柱时的温度为$T_3$, 则:
$p_1=$700毫米汞柱,$T_1=273$开;$P_2$是未知数,$T_2=30+273=303$开;$p_3=650$毫米汞柱,$T_3$是未知数,由于体积不变,根据查理定律
\[\frac{p_1}{p_2}=\frac{T_1}{T_2}\]
得:
\[p_2=\frac{T_2p_1}{T_1}=\frac{303\x 700}{273}=777{\rm mmHg}\]
又$\dfrac{p_1}{p_3}=\dfrac{T_1}{T_3}$
得:
\[T_3=\frac{p_3T_1}{p_1}=\frac{650\x 273}{700}=254{\rm K}=-19^{\circ}{\rm C}\]
\end{solution}
\item 一定质量的某种气体,在20$^\circ$C时的压强是$1.0\times 10^5$帕,如果保持它的体积不变,温度升高到50$^\circ$C时,它的压强是多大?温度降低到$-7^\circ$C时,它的压强又是多大?

\begin{solution}
 这一定质量的某种气体的三个状态,其压强和温度分别为
\begin{center}
\begin{tabular}{p{.3\textwidth}p{.3\textwidth}}
    $p_1=1.0\x10^5$帕 & $T_1=20^{\circ}{\rm C}=293{\rm K}$\\
    $p_2$是未知数& $T_2=50^{\circ}{\rm C}=323{\rm K}$\\
    $p_3$是未知数& $T_3=-7^{\circ}{\rm C}=266{\rm K}$\\
\end{tabular}
\end{center}
由于气体体积不变,根据查理定律$\dfrac{p_1}{p_2}=\dfrac{T_1}{T_2}$,得:
\[p_2=\frac{T_2p_1}{T_1}=\frac{323\x 1.0\x 10^5}{293}=1.1\x 10^5{\rm Pa}\]
又$\dfrac{p_1}{p_3}=\dfrac{T_1}{T_3}$,
\[p_3=\frac{T_3p_1}{T_1}=\frac{266\x 1.0\x 10^5}{293}=9.1\x 10^4{\rm Pa}\]
\end{solution}
\item 盛有氧气的钢筒,在室内(室温是17$^\circ$C)测得筒内气体的压强是$9.31\times 10^8$帕,当钢筒搬到温度是$-$13$^\circ$C的工地时,筒内气体的压强变为$8.15\times 10^8$帕.钢筒是不是漏了气?为什么?

\begin{solution}
先假定钢筒没有漏气,则气体满足查理定律$\dfrac{p_1}{p_2}=\dfrac{T_1}{T_2}$, 得
\[p_2=\frac{T_2p_1}{T_1}\]
其中$p_1=9.31\x10^6$帕,$T_1=17+273=290$开,$T_2=-13+273=260$开.
将数据代入上式得
\[p_2=\frac{260\x 9.31\x 10^6}{290}=8.35\x 10^6{\rm Pa}\]
而钢筒内实际气压为$p'_2=8.15\x 10^6$Pa,$p'_2<p_2$,故钢筒已漏气.
\end{solution}
\item 装在容器中的气体,体积为4升,压强为$2.0\times 10^5$帕,温度为300开,先让气体发生等容变化,压强增大为原来的2倍,然后让气体发生等温变化,压强又降低到原来的数值.求气体在末状态时的体积和温度.


\begin{solution}
根据题意,装在容器中的气体共经历两个变化过程,一个是等容变化过程,一个是等温变化过程,有三个状态,其参量分别为
\begin{center}
\begin{tabular}{p{.3\textwidth}p{.2\textwidth}p{.3\textwidth}}
$p_1=2.0\x10^5$帕&$V_1=4$升& $T_1=300$开\\
$p_2=4.0\x10^5$帕& $V_2=4$升&$T_2$为未知\\
$p_3=2.0\x10^5$帕& $V_3$为未知& $T_3=T_2$为未知  \\
\end{tabular}
\end{center}

根据查理定律
\[\frac{p_1}{p_2}=\frac{T_1}{T_2}\]
得
\[T_2=\frac{p_2}{p_1}T_1=\frac{4.0\x10^5}{2.0\x10^5}\x 300=600{\rm K}\]
根据玻意耳-马略特定律$p_2V_2=p_3V_3$, 得
\[V_3=\frac{p_2V_3}{p_3}=\frac{4.0\x10^5\x 4}{2.0\x10^5}=8{\rm L}\]
而$T_3=T_2=600$K.

气体末状态的体积是8升,温度是600开.
\end{solution}
\end{enumerate}

\subsection{练习四}

\begin{enumerate}
	\item 对一定质量的气体来说,能否做到:
	\begin{enumerate}
	\item	保持压强和温度不变而改变它的体积?
	\item	保持温度和体积不变而改变它的压强?
	\item	保持体积和压强不变而改变它的温度?	 
	\end{enumerate}

\begin{solution}
对于一定质量的气体,在状态变化过程中,它的三个状态参量始终应满足$\dfrac{pV}{T}=$恒量.如果任意改变了一个状态参量,而其余的两个参量一个也不发生相应的变化,那么$\dfrac{pV}{T}$不再等于原先的结果,这是违反气态方程的.因而题中所说的只改变一个状态参量的三种情况都是不可能做到的.
\end{solution}
\item  对一定质量的气体来说,能否做到:
\begin{enumerate}
\item 保持压强不变,同时升高温度并减小体积?
\item 保持温度不变,同时增加体积并减小压强?
\item 保持体积不变,同时增加压强并降低温度?
\end{enumerate}

\begin{solution}
\begin{enumerate}
    \item 不能做到.根据盖·吕萨克定律,在保持压强不变的情况下,一定质量的气体的体积应随温度的升高而增大;
    \item 能做到.根据玻意耳-马略特定律,在保持温度不变的条件下,
    一定质量的气体体积增大,压强必减小;
    \item 不能做到,根据查理定律,在体积不变的情况下,一定质量的气体的压强应随温度的降低而减小.
\end{enumerate}
\end{solution}
\item  一定质量的空气,27$^\circ$C时的体积为$1.0\times 10^{-2}{\rm m}^3$.计算在压强不变的情况下,温度升高到100$^\circ$C时的体积.

\begin{solution}
气体在状态1时:$T_1=273+27=300$开,$V_1=1.0\x10^{-2}{\rm m^3}$

气体在状态2时:$T_2=273+100=373$开,$V_2$为待求体积.

由盖·吕萨克定律$\dfrac{V_1}{V_2}=\dfrac{T_1}{T_2}$得到
\[V_2=\frac{T_2}{T_1}V_1=\frac{373}{300}\x 1.0\x 10^{-2}{\rm m^3}=1.2\x 10^{-2}{\rm m^3}\]
\end{solution}
\item  某种柴油机的气缸容积为$0.83\times 10^{-3}{\rm m^3}$.压缩前其中空气的温度为47$^\circ$C,压强为$0.8\times 10^5$帕,在压缩冲程,活塞把空气压缩到原体积的1/17,压强增大到$40\times 10^5$帕.求这时空气的温度.

\begin{solution}
    气体压缩前为初状态:
\[V_1=0.83\x10^{-3}{\rm m^3},\quad T_1=273+47=320{\rm K},\quad p_1=0.8\x10^5{\rm Pa}\]
压缩后为末状态:
\[V_2=\frac{1}{17}V_1,\qquad  p_2=40\x10^5{\rm Pa}\]
$T_2$为待求温度,由气态方程$\dfrac{p_1V_1}{T_1}=\dfrac{p_2V_2}{T_2}$得到
\[T_1=\frac{40\x10^5\x320}{0.8\x10^5\x17}=941{\rm K}=668^{\circ}{\rm C}\]
\end{solution}
\item  在容积为25升的容器中,盛有温度为37$^\circ$C、压强为62标准大气压的氧气.求氧气在标准状态(0$^\circ$C,1标准大气压)下的体积,从化学课中学过,在标准状态下,1摩的任何气体的体积都是22.4升.你能不能由此求得容器中氧气的摩尔数并进而求得氧气的质量?怎样求?


\begin{solution}
氧气在初状态时:$V_1=25$升,$T_1=273+37=310$开,$p_1=62$标准大气压;在末状态即标准状态时:$T_0=273$开,
$p_0=1$标准大气压,$V'_0$为待求体积.由气态方程
\[\frac{p_1V_1}{T_1}=\frac{p_0V'_0}{T_0}\]
得到
\[V'_0=\frac{p_1V_1}{T_1}\x \frac{T_0}{p_0}=\frac{62\x 25}{310}\x \frac{273}{1}=1.4\x 10^3{\rm L}\]
$V'_0$为$n$摩的氧气在标准状态下占有的体积,有$V'_0=nV_0$,故
\[n=\frac{V'_0}{V_0}=\frac{1.4\x 10^3}{22.4}=63{\rm mol}\]
又因为$n=m/M$, 氧气的摩
尔质量$M=32\x10^{-3}$千克/摩,所以摩尔数为$n$的氧气质量
$m=n\x M=63x32x10^{-3}=2.0$千克.
\end{solution}
\item  一个瓶子里装有某种气体,瓶上有一个小孔跟外面大气相通.原来瓶里气体的温度为15$^\circ$C.如果把它加热到207$^\circ$C,瓶里保留的气体的质量是原来质量的几分之几?

\begin{solution}
 瓶中气体状态变化为一等压过程,$p_1=p_2=p_0$, $T_1=273+15=288$开,$V_1=V_{\text{气}}$, $T_2=273+207=480$开,$V_2$待求,由盖·吕萨克定律.$\dfrac{V_1}{T_1}=\dfrac{V_2}{T_2}$,得到
\[V_2=\frac{V_1}{T_1}T_2=\frac{480}{288}V_{\text{气}}\]
又因为质量$m$, 密度$D$和体积$V$有关系$m=D\x V$, 所以
\[\frac{m_{\text{气}}}{m}=\frac{DV_{\text{气}}}{DV_2}=\frac{288}{480}=0.6=60\%\]
即瓶里保留气体的质量是原来质量的60\%.
\end{solution}
\item  贮气筒内装有压缩气体,温度是27$^\circ$C,压强是$40\times 10^5$帕,如果从筒内放出一半质量的气体,并使筒内剩余的气体的温度降到12$^\circ$C,这些剩余气体的压强是多大?

\begin{figure}[htp]
    \centering
\includegraphics[scale=.8]{fig/3-8.png}
    \caption{}
\end{figure}

\begin{solution}
如图3.8所示,我们把筒中一半质量$m/2$的气体作为
    研究对象,由题意$T_1=273+27=300$开,$p_1=40\x10^5$帕,$V_1=V/2$; $T_1=273+12=285$开,$V_2=V$, $p_2$待求.
    由, $\dfrac{p_1V_1}{T_1}=\dfrac{p_2V_2}{T_2}$得到
\[p_2=\frac{p_1V_1}{T_1}\x \frac{T_2}{V_2}=\frac{10\x 10^5\x \frac{V}{2}}{300}\x \frac{285}{V}=1.9\x 10^6{\rm Pa}\]
\end{solution}
\end{enumerate}

\subsection{练习五}

\begin{enumerate}
	\item 如果压强用标准大气压作单位,体积用升作单位,试通过计算证明:$R=0.082$标准大气压·升/(摩·开).

\begin{solution}
\[\begin{split}
    R=\frac{p_0V_0}{T_0}&=\frac{1\text{标准大气压}\x 22.4\text{升/摩}}{273\text{开}}\\
    &=0.082\text{标准大气压·升/(摩·开)}
\end{split}\]
\end{solution}
\item 一个容器内装有氧气100克,压强为10标准大气压,温度为47$^\circ$C,容器的容积是多少${\rm m^3}$?

\begin{solution}
    根据克拉珀龙方程$pV=\dfrac{m}{M}RT$,得到
\[V=\frac{mRT}{Mp}=\frac{100\x 0.082\x (47+273)}{32\x 10}=8.2{\rm L}=8.2\x 10^{-3}{\rm m^3}\]
该题中$R$取值$8.31{\rm J/(mol\cdot K)}$,也可得出相同的结果.
\end{solution}
\item 1克的气体,温度为27$^\circ$C、压强为600毫米汞柱时,体积为5升,2克的同种气体,温度为127$^\circ$C、压强为400毫米汞柱时,体积是多少升?

\begin{solution}
    根据克拉珀龙方程$pV=\dfrac{m}{M}RT$得到$V=\dfrac{mRT}{Mp}$,则
\[V_1=\frac{m_1RT_1}{Mp_1},\qquad V_2=\frac{m_2RT_2}{Mp_2}\]
\[\frac{V_2}{V_1}=\frac{m_2T_2p_1}{m_1T_1p_2}=\frac{2\x (127+273)\x 600}{1\x (27+273)\x 400}=4\]
$\therefore\quad V_2=4V_1=4\x 5=20{\rm L}$
\end{solution}
\item 容积是10升的钢筒里盛有90标准大气压、$-13^\circ$C的氧气,求钢筒中氧气的质量.已知氧气在标准状态下的密度$\rho_0=1.43{\rm kg}/{\rm m^3}$.

\begin{solution}
由题意知:$V_1=10$升,$p_1=90$标准大气压,$T=273-13=260$开,$M=32\x10^{-3}$千克/摩.由克拉珀龙方程$pV=\dfrac{m}{M}RT$得到
\[m=\frac{pVM}{RT}=\frac{90\x10\x32\x10^{-3}}{0.082\x260}=1.4{\rm kg}\]

    或者根据定质量气态方程求出标准状态下的体积$V'_0$由$\dfrac{p_1V_1}{T_1}=\dfrac{p_0V'_0}{T_0}$得
\[V'_0=\dfrac{p_1V_1}{T_1}\x \frac{T_0}{p_0}=\frac{90\x 10}{260}\x\frac{273}{1}=9.5\x 10^2{\rm L}\]    
又$m=\rho_0 V'_0=1.43\x0.95=1.4$千克,结果相同.
\end{solution}
\item 有0.612克的某种氮氧化合物,在293开和1标准大气压时体积为480${\rm cm^3}$,这是一种什么气体?写出它的分子式.

\begin{solution}
         求出该气体的摩尔质量,即可知道是什么气体.根据克拉珀龙方程$pV=\dfrac{m}{M}RT$得到
\[M=\frac{mRT}{pV}=\frac{0.612\x 10^{-3}\x 0.082\x 293}{1\x 480\x 10^{-3}}=0.030{\rm kg}=30{\rm g}\]
由于氮原子的摩尔质量是14克,氧原子的摩尔质量是16克,所以这种气体是一氧化氮,分子式为NO.
\end{solution}
\item 给汽车轮胎打气,使胎内空气达到所需的压强,冬天和夏天打入胎内的空气质量是否相同?冬天还是夏天打入的空气质量多?

\begin{solution}
由克拉珀龙方程$pV=\dfrac{m}{M}RT$可知,当$p$、$V$一定时,$m$与$T$成反比,要达到一定的压强$p$, 冬天温度$T$低,打入胎内的空气质量多一些.
\end{solution}
\item 有两种不同种类的气体,它们的温度和体积都相同.如果它们的质量也相同,气体的压强是否相同?如果它们的质
量不同,但摩尔数相同,气体的压强是否相同?

\begin{solution}
根据克拉珀龙方程$pV=\dfrac{m}{M}RT$, 当$T_1=T_2$, $V_1=V_2$, $m_1=m_2$时,因$M_1\ne M_2$, 所以$p_1\ne p_2$. 又因为$\dfrac{m}{M}=n$,所以克拉珀
龙方程也可写为$pV=nRT$.当$T_1=T_2$, $V_1=V_2$, $n_1=n_2$时,自然有$p_1=p_2$.
\end{solution}
\item 理想气体的状态方程可写成$pV/T=C$(恒量),对于这个恒量$C$,下面哪种说法正确,哪种说法错误,并说明理由.
\begin{enumerate}
\item 对质量相同的任何气体,$C$都相同.\item 对质量不同的同种气体,$C$都相同.	\item 对摩尔数不同的同种气体,$C$都相同.\item 对摩尔数相同的任何气体,$C$都相同.
\end{enumerate}

\begin{solution}
根据克拉珀龙方程有$\dfrac{pV}{T}=\dfrac{m}{M}R=C$. 当气体质量$m$相同,而摩尔质量$M$不同时,$C$不同.同种气体$M$相同,$m$不同时,$C$也不同.当把克拉珀龙方程写为$\dfrac{pV}{T}=nR=C$时,可以很容易的判断:对摩尔数$n$不同的同种气体$C$不同,而对摩尔数相同的任何气体$C$都相同,所以上述说法中只有(d)正确.
\end{solution}
\end{enumerate}

\subsection{练习六}

\begin{enumerate}
	\item 现在我们用另一种方法估算一下气体分子间的距离与分子直径的关系.在标准状态下,1摩的气体占有22.4升的体积.我们设想其中的每个分子都位于一个小立方体的中心.这个小立方体的边长是多少?分子直径的数量级为$10^{-10}$米.把小立方体的边长跟分子直径相比较,结果怎样?

\begin{solution}
    小立方体的边长
\[a=\sqrt[3]{V_1}=\sqrt[3]{\frac{V_0}{N}}=\sqrt[3]{\frac{22.4\x 10^{-3}}{6.02\x 10^{23}}}=3.34\x 10^{-9}{\rm m}\]
把小立方体边长$a$与分子直径$d$的数量级相比较:
\[\frac{a}{d}=\frac{3.34\x 10^{-9}}{10^{-10}}\approx 30\]
即在标准状态下,气体分子之间的
距离约为分子直径的30倍.
\end{solution}
\item 根据第七节表中的数据能不能估算出0$^\circ$C和100$^\circ$C时氧气分子的平均速率?怎样估算?结果怎样?

\begin{solution}
    可以估算,将各区间的速率大小取中值,比如:100—
200(米/秒)取150(米/秒),500—600(米/秒)取550(米/秒)
等等,900米/秒以上的取950米/秒.再分别乘以各速率区间的
分子数占总分子数的百分率,最后除以100即可得平均速率.

在$0^{\circ}{\rm C}$时氧气分子的平均速率
\[\begin{split}
    \bar v&=\frac{1}{100}(50\x1.4+150\x8.1
+250\x17.0+350\x21.4+450\x20.4\\
&\qquad +550\x15.1+650\x9.2
+750\x4.5+850\x2.0+950\x0.9) \\
&=424{\rm m/s}
\end{split}\]

同理可求得$100^{\circ}{\rm C}$时氧气分子的平均速率为494{\rm m/s}.
\end{solution}
\end{enumerate}



\subsection{练习七}
\begin{enumerate}
	\item 一定质量的理想气体在温度不变的情况下被压缩,气体的内能是否改变?外界对气体是否做功?气体从外界吸热还是向外界放热?功和热量有什么关系?

    \begin{solution}
        由于温度不变,所以气体的内能不变;气体被压缩时
外界对气体作功;根据热力学第一定律$\Delta E=W+Q$知气体向
外界放热;功和热量相等.
    \end{solution}
\item 一定质量的理想气体在体积不变的情况下压强减小,这时外界对气体是否做功?气体的内能是否改变,怎样改变?气体放出的热量跟内能的改变有什么关系?

\begin{solution}
    由于气体的体积不变,外界不对气体作功;气体的体
积不变,压强减小时,温度要降低,因而气体的内能要减少;根
据热力学第一定律气体放出的热量跟内能的减少相等.
\end{solution}
\item 一定质量的理想气体在压强不变的情况下体积减小,外界对气体是否做功?气体的内能是否改变,怎样改变?气体放热还是吸热?这个热量跟内能的改变有什么关系?

\begin{solution}
    在压强不变的情况下体积减小,外界对气体作了功;
等压变化时,体积减小,温度将降低,故气体内能减少;由热力
学第一定律知气体将放热,这个热量等于外界对气体所做的
功跟气体减少的内能之和.
\end{solution}
\end{enumerate}

\subsection{习题}
\begin{enumerate}
	\item 下面几种说法,哪个正确,哪个错误,并说明理由.
	\begin{enumerate}
		\item 有两个相同的容器,内装同种气体,它们的压强相同,因而它们的温度一定相同.
		\item 有两个相同的容器,内装质量相同的不同气体,它们的压强不同,因而它们的温度一定不同.
		\item 有两个相同的容器,内装摩尔数相同的气体,它们的压强相同,因而它们的温度一定相同.
	\end{enumerate}

\begin{solution}
\begin{enumerate}
    \item 这种说法是错误的.根据克拉珀龙方程$pV=\dfrac{m}{M}RT$得到$T=\dfrac{pVM}{mR}$,题意表明,两容器中的气体的$M$、$p$、$V$相同,但气体的质量不一定相同,所以说“它们的温度一定相同”
    是错误的.
    \item 这种说法也是错误的.由$T=\dfrac{pVM}{mR}$,
    根据题意其中$m$、$V$是相同的,$R$是恒量,$M$不同,$p$也不同,因此存在两气
    体$p\cdot M$的积相同的可能性,即存在温度相同的可能性,所以
    说“它们的温度一定不同”是错误的.
    \item 这种说法是正确的.由克拉珀龙方程$pV=nRT$, 得$T=\dfrac{pV}{nR}$,
    依题意$p$、$V$、$n$相同,$R$是恒量,所以$T$一定相同.
\end{enumerate}
\end{solution}
\item 一定质量的理想气体,处在某一初始状态,现在要使气体的温度经过状态变化后回到初始状态的温度,用下列哪些过程可能实现?
\begin{enumerate}
	\item 先保持压强不变而使它的体积膨胀,接着保持体积不变而减小压强.
\item 先保持压强不变而使它的体积减小,接着保持体积不变而减小压强.
\item 先保持体积不变而增大压强,接着保持压强不变而使它的体积膨胀.
\item 先保持体积不变而减小压强,接着保持压强不变而使它的体积膨胀.	 

\begin{solution}
    由于这一定质量的理想气体初末状态的温度相同,根
    据$\dfrac{pV}{T}=$恒量,其压强与体积将成反比,因此,初末状态温度
    相等的过程,其末状态可能是压强变大、体积变小或压强变
小,体积变大,也可以是压强、体积都不变,末状态的压强和体
积同时变大或同时变小的过程是不能实现的,所以(a)、(d)所
述的过程是可以实现的.
\end{solution}
\end{enumerate}
\item 盖·吕萨克定律如果用摄氏温标$t$来表示,可以写成下式:
\[V_t=V_0\left(1+\frac{t}{273}\right) \]
其中$V_0$和$V_t$分别表示气体在0$^\circ$C和t$^\circ$C时的体积.试推导出上式.

\begin{solution}
    若一定质量的气体作等压变化,在$0^{\circ}{\rm C}$时的体积为
    $V_0$, 在$t^{\circ}{\rm C}$时的体积为$V_t$, 由盖·吕萨克定律有
\[\frac{V_t}{V_0}=\frac{T_t}{T_0}\]
即\[V_t=\frac{T_t}{T_0}V_0\]
其中:
\[T_t=(t+273){\rm K},\qquad T_0=273{\rm K}\]

将$T_t$、$T_0$值代入上式,得到
\[V_t=\frac{t+273}{273}V_0\]
即
\[V_t=V_0\left(1+\frac{t}{273}\right) \]


\end{solution}
\item 能不能根据玻意耳-马略特定律和盖·吕萨克定律推出一定质量的理想气体的状态方程$pV/T=$恒量?实际推导一下.

\begin{figure}[htp]
    \centering
\includegraphics[scale=.8]{fig/3-9.png}
    \caption{}
\end{figure}

\begin{solution}
    能推导.如图3.9所示,设容器内密闭了一定质量的气体,其状
    态参量为$p_1$、$V_1$、$T_1$, 先经过一个等温变化过程,使气体达到
    一个中间状态,其状态参量为$p_2$、$V_c$、$T_1$, 根据玻意耳-马略特
    定律得到
\begin{equation}
    p_1V_1=p_2V_c
\end{equation}

再经过一个等压变化过程,状态参量变为$p_2$、$V_2$、$T_2$, 根
据盖·吕萨克定律得到
\begin{equation}
    \frac{V_c}{V_2}=\frac{T_1}{T_2}
\end{equation}
将(3.1)代入(3.2)式得
\[\frac{p_1V_1}{p_2V_2}=\frac{T_1}{T_2}\]
即
\[\frac{p_1V_1}{T_1}=\frac{p_2V_2}{T_2}\]
上式说明,一定质量的气体从初状态$(p_1,V_1,T_1)$变到末
状态$(p_2,V_2,T_2)$, 压强和体积的乘积与热力学温度的比值是
不变的.即
\[\frac{pV}{T}=\text{恒量}\]
\end{solution}
\item 当温度为27$^\circ$C、压强为$2.0\times 10^5$帕时,32克氧气的体积是多大?密度是多大?另有48克氧气,温度和压强跟上述数值相同,氧气的密度又是多大?	

\begin{solution}
    根据克拉珀龙方程$pV=\dfrac{m}{M}RT$得到
\[\begin{split}
    V&=\frac{mRT}{Mp}=\frac{32\x 8.31\x (27+273)}{32\x 2.0\x 10^5}=13\x 10^{-3}{\rm m^3}=13{\rm L}\\
    \rho&=\frac{m}{V}=\frac{32}{13}=2.5{\rm g/L}
\end{split}\]

由克拉珀龙方程可得
\[\rho=\dfrac{m}{V}=\dfrac{pM}{RT}\]

由上式可知,$\rho$与质量$m$无关,由于该题条件给出$M$、$p$、$T$
都相同.$R$是恒量,故$m=48$克时,$\rho$值不变,即
$\rho=2.5{\rm g/L}$.
\end{solution}
\item   试根据克拉珀龙方程推导出用压强和温度来表示的气体密度的表达式.

\begin{solution}
    根据克拉珀龙方程$pV=\dfrac{m}{M}RT$,

    $\therefore\quad \rho=\dfrac{m}{V}=\dfrac{Mp}{RT}$
\end{solution}
\item  水银气压计中混入了一个空气泡,上升到水银柱的上方,使水银柱上方不再是真空,因而气压计的读数比实际的大气压小些,当实际大气压为768毫米汞柱时,气压计的读数只有750毫米汞柱,此时管中水银面到管顶的距离为80毫米,当气压计读数为740毫米汞柱时,实际大气压为多少?设温度保持不变.

\begin{figure}[htp]
    \centering
\includegraphics[scale=.6]{fig/3-10.png}
    \caption{}
\end{figure}

\begin{solution}
    根据题设条件作出示意图,如图3.10所示.当气压计
读数为750毫米汞柱时(甲图),水银气压计上端所密闭的空
气的压强$p_1$, 体积为$V_1$.
\[\begin{split}
    p_1&=768-750=18{\rm mmHg}\\
    V_1&=80\x S
\end{split}\]
其中$S$为玻璃管横截面积.

当气压计读数为740毫米汞柱时(乙图),上端密闭的空
气的压强为$p_2$, 体积为$V_2$, 其中
\[V_2=(80+750-740)\x S=90\x S\]

因为温度不变,由玻-马定律$p_1V_1=p_2V_2$得到
\[p_2=\frac{V_1}{V_2}p_1=\frac{80S}{90S}\x 18=16{\rm mmHg}\]

故此时实际大气压强$p_0=p_2+740{\rm mmHg}$
即:
\[p_0=16+740=756{\rm mmHg}\]

\end{solution}
\item  在湖面下50米深处(温度为4$^\circ$C)有一个体积为10
${\rm cm^3}$的气泡升到湖面上来,湖面的温度为17$^\circ$C,求它升到湖面时的体积,大气压强为$1.013\times 10^5$帕.

\begin{solution}
    对气泡内的气体,初状态时:$V_1=10{\rm cm^3}$, $T_1=273
+4=277{\rm K}$,$p_1=p_o+\rho gh=1.013\x10^5+10^3\x9.8\x50=
5.9\x10^5{\rm Pa}$.

末状态时:$T_2=273+17=290{\rm K}$,$p_2=1.013
\x10^5{\rm Pa}$,$V_2$待求.

根据气态方程$\dfrac{p_1V_1}{T_1}=\dfrac{p_2V_2}{T_2}$得到
\[V_2=\frac{p_1V_1T_2}{T_1p_2}=\frac{5.9\x10^5\x10\x290}{277\x1.013\x10^5}{\rm cm^3}=61{\rm cm^3}
\]
\end{solution}
\item  有两个容积相等的器,里面盛有同种气体,用一段水平玻璃管把它们连结起来.在玻璃管的正中央有一段水银柱,当一个容器中气体的温度是0$^\circ$C,另一个容器中气体的温度是20$^\circ$C时,水银柱保持静止,如果使两容器中气体的温度都升高10$^\circ$C,管中的水银柱会不会移动?如果移动的话,向哪个方向移动?试根据学过的气体定律加以说明.

\begin{solution}
    先假定两边密闭容器中的气体体积暂不改变,根据查
    理定律.分别计算出两边气体升温10$^\circ$C后的压强,再比较两
    方压强的大小,就能判断水银柱会不会移动和向哪个方向
    移动.

    对原来的温度为0$^\circ$C的气体:
\[\frac{T_1}{T_2}=\frac{p_1}{p_2},\qquad p_2=\frac{T_2}{T_1}p_1=\frac{283}{273}p_0=1.04p_0\]
    对原来的温度为20$^\circ$C的气体:
\[\frac{T'_1}{T'_2}=\frac{p'_1}{p'_2},\qquad p'_2=\frac{T'_2}{T'_1}p'_1=\frac{303}{293}p_0=1.03p_0\]
此时$p_2>p_2'$, 所以水银柱向原来温度高的那一侧移动.
\end{solution}
\item  一个容器,如果其中气体十分稀薄,通常就说这个容器为“真空”.有一个容积为10${\rm cm^3}$的电子管,在温度为300开时用真空泵把它抽成真空,使管内气体压强为 $5\times 10^{-8}$毫米汞柱,这时管内有多少个气体分子?

\begin{solution}
    根据克拉珀龙方程$pV=nRT$, 得到
\[n=\frac{pV}{RT}\]
其中$p=5\x10^{-7}{\rm cmHg}=6.6\x10^{-9}$标准大气压,
\[V=10{\rm cm^3}=0.010{\rm L},\qquad T=300{\rm K}\]
代入上式
\[n=\frac{6.6\x 10^{-9}\x 0.010}{0.082\x 300}=2.7\x10^{-12}{\rm mol}\]
所含分子数
\[N=nN_0=2.7\x10^{-12}\x6.02\x10^{23}=1.6
\x10^{12}\text{个}\]
\end{solution}
\item  氧气瓶的容积是32升,其中氧气的压强是130标准大气压.规定瓶内氧气压强降到10标准大气压时就要重新充氧.有一个车间,每天需用1标准大气压的氧气400升.这瓶氧气能用几天?假定温度保持不变.

\begin{solution}
氧气瓶内压强下降时,因温度保持不变,可应用玻-马
定律有$p_1V_1=p_2V_2$, 得32升、压强为130标准大气压的氧气,
压强降到10标准大气压时的体积
\[V_2=\frac{p_1V_1}{p_2}=\frac{130\x32}{10}=416{\rm L}\]

能放出供使用的压强为10标准大气压的氧气的体积
\[V'=V_2-V_1=416-32=384{\rm L}\]

384升、10标准大气压的氧气使用时降为1标准大气压
后的体积
\[V''=\frac{p_2V'}{p_0}=\frac{10\x 384}{1}=3840{\rm L}\]

每天用400升、一标准大气压的氧.则可用天数
\[n=\frac{3840}{400}=9.6\text{天}\]
即这瓶氧气能用9.6天.
\end{solution}
\item  如图3.11所示,气缸$A$和容器$B$由一细管经阀门
$K$相联.$A$和$B$的壁都是透热的.$A$放在27$^\circ$C、1标准大气压的大气中,$B$浸在127$^\circ$C的恒温槽内.开始时$K$是关断的,
$B$内没有气体,容积$V_B=2.4$升;$A$内装有气体,体积$V_A=
4.8$升.打开$K$,使气体由$A$流入$B$,等到活塞$D$停止移动
时,$A$内气体的体积是多大?假设活塞$D$与气缸壁之间没有摩擦,细管的容积忽略不计.

\begin{figure}[htp]\centering
	\begin{tikzpicture}[>=stealth, scale=.5 ]


\draw[very thick] (-2, -1.5)--(2, -1.5);
\draw[very thick] (-2, 1.5)--(2, 1.5);
\draw[very thick] (2,.2)--(5,.2); \draw[very thick] (2,-.2)--(5,-.2);
\draw[very thick] (5, 2.5)--(10, 2.5); \draw[very thick] (5, -2.5)--(10, -2.5);
\draw[very thick] (-2, 1.5)--(-2, -1.5);
\draw[very thick] (2, 1.5)--(2, .2);
\draw[very thick] (2, -.2)--(2, -1.5);
\draw[very thick] (5, 2.5)--(5, .2);
\draw[very thick] (5, -.2)--(5, -2.5);
\draw[dashed](-3,-2.5) rectangle (3,2.5);

\fill [pattern=north east lines, draw](8,-2.5) rectangle(8.5,2.5);
\node at (0,2){127$^\circ$C}; \node at (9,2){$D$}; 
\node at (0,0){$B$};  \node at (7,0){$A$};

\draw [->](10,0)node [below]{$p_0$}--(8.5,0);

\draw(3.5,-.5)--(3.5,.5);
\draw(3.2,.5)--(3.7,.5)node[right]{$K$};

\end{tikzpicture}
\caption{}
\end{figure}

\begin{solution}
    开关$K$打开前只有容器$A$中有气体,设其质量为$m_0$.
打开$K$后,$A$中有部分气体流入容器$B$, 设其质量分别为$m_A$
及$m_B$, 且$m_0=m_A+m_B$. 根据克拉珀龙方程
$pV=\dfrac{m}{M}RT$,得到
\[m=\frac{MpV}{RT}\]
考虑到以上的质量关系,得到
\[\frac{Mp_AV_A}{RT_A}=\frac{Mp'_AV'_A}{RT_A}=\frac{Mp_BV_B}{RT_B}\]
其中$p_A=p'_A=p_B=p_0$, $V_A=4.8$升,
$T_A=27+273=300$开,
$T_B=127+273=400$开,$V_B=2.4$升.

上式变成
\[\frac{V_A}{T_A}=\frac{V'_A}{T_A}+\frac{V_B}{T_B}\]
$\therefore\quad V'_A=V_A-\dfrac{T_A}{T_B}V_B=\left(4.8-\dfrac{300}{400}\x2.4\right)=3.0{\rm L}$
\end{solution}
\end{enumerate}

\section{参考资料}
\subsection{压强的单位及换算关系}
\begin{enumerate}
    \item 帕(Pa):为压强的国际单位制单位,$1{\rm Pa}=1{\rm N/m^2}$
    \item 巴(bar):为压强在厘米克秒制中的单位,$1{\rm bar}=1{\rm dyn/cm^2}$.
    \item 厘米汞柱(cmHg):1厘米高汞柱产生的压强.
    \item 标准大气压(atm):等于76厘米高汞柱产生的压强.
    \item 工程大气压(at):$1{\rm at}=1{\rm kg/cm^2}$
\end{enumerate}

压强单位的换算关系:
\begin{enumerate}
    \item $1{\rm Pa}=1{\rm N/m^2}=10^5{\rm dyn/10^4 cm^2}=10{\rm bar}$
    \item $1{\rm atm}=76{\rm cmHg}=1033.6{\rm g/cm^2}=1.013\x 10^6{\rm bar}=1.013\x 10^5{\rm Pa}$
    \item $1{\rm at}=0.98\x 10^6{\rm bar}=0.98\x 10^5{\rm Pa}$    
\end{enumerate}

说明:国务院颁布的法定计量单位中规定,帕为法定计量
单位中的国际单位制单位,其余的都是应废除的计量单位.

\subsection{温度的单位及换算关系}
华氏度:1714年德国物理学家海伦海特制成水银温度
计,建立了华氏温标,这种温标是用水银作测温介质,以水银
在玻璃容器内相对膨胀来表示温度,把标准大气压下冰的熔
点定为32度,水的沸点定为212度,两点之间等分为180格,
每一格称为1华氏度,用$^{\circ}{\rm F}$表示.

摄氏度:1742年瑞典天文学家摄尔西乌斯用与上述相同
的温度计,也把标准大气压下冰的熔点和水的沸点作为原始
分度点,建立了摄氏温标.这两个原始分度点的温度分别定
为0度和100度,两点之间等分100格,每一格称为1摄氏
度,用$^{\circ}{\rm C}$表示.

开尔文:为温度的国际单位制单位,几经演变,于1967—1968年第十三届国际计量大会确认:1开尔文等于水的三相
点热力学温度的1/273.16.(1954年第十届国际计量大会曾
作出决议:以水的三相点为基本定点定义热力学温标,并指定
它的温度为273.16开尔文.)

说明:水的三相点的温度为$0.01^{\circ}{\rm C}$, 压强为4.58毫米
汞柱.

三种温标的分度点及分度情况如图3.12所示
三种温度的换算关系:
\begin{figure}[htp]
    \centering
    \includegraphics[scale=.8]{fig/3-12.png}
    \caption{}
\end{figure}

\[\begin{split}
   T({\rm K}) &= t({\rm ^{\circ}C})+273.15\approx t({\rm ^{\circ}C})+273\\
   t_{\rm F}({\rm ^{\circ}F})&=\frac{9}{5}t({\rm ^{\circ}C})+32 
\end{split}\]

\subsection{摩尔}
摩尔是物质的量的单位,它是由以往的“克分子”演变而
来的.1971年第十四届国际计量大会正式通过,将“摩尔”定
为国际单位制的第七个基本单位.它的定义是:“构成物质系
统的结构粒子数目和0.012kg碳-12中的原子数目相等,则这
个系统的物质的量为1摩尔,单位符号为mol, 在使用摩尔
时,结构粒子应予指明,可以是原子、分子、离子、电子及其他
粒子,或是这些粒子的特定组合.”

由上述定义可见,摩尔具有双重含义,它既表示一种物质
的质量,又能表示该物质所包含的结构粒子数.

0.012kg碳-12中的原子数为阿佛伽德罗常数$N_0$, $N_0=6.022\x10^{23}/{\rm mol}$,因此,1摩尔的任何原子、分子、离子、电子
及其他粒子的数目皆为$N_0$.

对于由分子组成的物质系统,该物质1摩尔就是$N_0$个
分子的集合,若以克/摩为单位,该物质的摩尔质量恰好等于
物质的分子量.如氢的分子量为2.02, 氢的摩尔质量为2.02
克/摩或$2.02\x10^{-3}$千克/摩.

\subsection{平衡态}
教材中提到的“气体处于一定的状态中”,指的是平衡态.

在没有外界影响条件下的热力学系统,经过一定时间后,
将达到一个确定的状态,而不再有任何宏观变化,这种在不
受外界影响的条件下,宏观性质不随时间变化的状态叫做平
衡态,这里所说的没有外界影响,是指外界对系统既不作功又
不传热.如果系统通过作功或传热的方式与外界交换能量,
则它就不可能达到并保持在平衡态,当然,在实际中并不存
在完全不受外界影响,而且宏观性质绝对保持不变的系统,
所以平衡态是一种理想化的状态,它是在一定条件下对实际
情况的概括和抽象.

应当指出,平衡态是指系统的宏观性质不随时间变化.从
微观方面看,在平衡态下,组成系统的分子仍在不停地运动
着,只不过分子运动的平均效果不随时间改变,而这种平均效
果的不改变在宏观上就表现为系统达到了平衡态.因此,热
力学中的平衡是动平衡,通常特别把这种平衡叫做热动平衡.

\subsection{实际气体不能严格遵守三个气体定律的原因}
精确的实验表明一切实际气体都只是近似地遵守玻意耳
-马略特定律、查理定律和盖·吕萨克定律.当气体压强不太大
(与大气压相比)、温度不太低(与室温比较)时,实际测量的结
果与上述定律得出的结果相差不大.当压强很大、温度很低
时,实际测量结果和由上述定律得出的结果有很大差别,现
在就气体的等温变化来研究这一问题,下表列举了几种常见
气体在$0^{\circ}{\rm C}$和不同压强下,压强和体积乘积$pV$的实验值.实验
气体在$0^{\circ}{\rm C}$、1标准大气压下的体积为1升.

\begin{center}
\begin{tabular}{c|cccc}
    \hline
    $p$ & \multicolumn{4}{|c}{$pV$(大气压·升)}\\
    \cline{2-5}
(标准大气压)&  ${\rm H}_2$&  ${\rm N}_2$&  ${\rm O}_2$&  空气\\
\hline
1&1.0000&1.0000&1.0000&1.0000\\
100&1.0690&0.9941&0.9265&0.9730\\
200&1.1380&1.0483&0.9140&1.0100\\
500&1.3565&1.3900&1.1560&1.3400\\
1000&1.7200&2.0685&1.7355&1.9920\\
\hline
\end{tabular}
\end{center}

为什么实际气体在压强很大时不能遵守玻意耳-马略特
定律呢?根据分子动论来分析,原因有二:

\subsubsection{分子间有相互作用力}
实际气体的分子间都有相互作用,除了分子相距很近表
现为斥力外,相距稍远时则表现为吸引力,距离再大,超过几
百埃时,则相互作用力趋于零.

当压强较小时,气体分子间距离较大,分子间相互作用力
以不计,因此实际气体的性质近似于理想气体,但当压强
很大时,分子间距离变小,分子间的相互吸引力增大.靠近器
壁的气体分子则受到向内的拉力,使分子在垂直于器壁方向
上的动量减小.因而气体对器壁的压强比不存在分子间的吸
引力时的压强要小.由于这一原因,使得实际气体当压强较
大时,实测的$pV$值有比由玻意耳-马略特定律计算出来的理
论值偏小的倾向.

\subsubsection{分子本身占有一定体积}

分子半径的数量级是$10^{-8}$厘米,按半径为$10^{-8}$厘米计
算,每个分子的体积是
\[\frac{4}{3}\pi r^3=4.2\x 10^{-24}{\rm cm^3}\]
1摩气体
在标准状态时体积是$22.4\x10^3{\rm cm^3}$, 分子个数为$6.022
\x10^{23}$. 故1摩气体的分子本身的体积是$$6.022\x10^{23}\x4.2
\x10^{-24}{\rm cm^3}\approx 2.5{\rm cm^3}$$

当压强较小时,分子本身体积约为气体体积的万分之一,
可以略去不计.因此实际气体的性质近似于理想气体,能遵守
玻意耳-马略特定律.当压强很大时,例如$p=1000$大气压,
假定玻意耳-略特定律仍能适用,则$0^{\circ}{\rm C}$的1摩的气体体积
将缩小为
\[\frac{22.4\x 10^3}{1000}=22.4{\rm cm^3}\]
分子本身体积为2.5${\rm cm^3}$, 大约是上述计算结果的
$1/10$,分子本身体积就不能不考虑
了.应当指出,气体能够压缩是分子间的距离减小了,而分子
本身体积则是不能压缩的.由于这一原因,实际气体当压强
较大时,实测的$pV$值有比计算出来的理论值偏大的倾向.

由于以上两个原因,一个是使气体的$pV$实验值偏大,一
个是使气体的$pV$的实验值偏小,这两个原因中,哪一个原
因占优势,就在哪一方面发生偏离.这就是实际气体在压强
很大时不能严格遵守玻意耳-马略特定律的原因.

同样,盖·吕萨克定律和查理定律用于实际气体也有偏
差.任何一种气体的体胀系数和压强系数并不总是$1/273$,
都随温度略有变化.各种气体的体胀系数与压强系数相比也
都略有差异,如下表所示:
\begin{center}
    \begin{tabular}{ccc}
\hline
气体  & 体胀系数&压强系数\\
\hline
氢&0.0036600&0.0036613\\
氦&0.0035820&0.0036601\\
氮&0.0036732&0.0036744\\
二氧化碳&0.0037414&0.0037266\\
空气&0.0036760&0.0036750\\
\hline
    \end{tabular}
\end{center}
其中的气体处于温度是$0^{\circ}{\rm C}$,
压强是76cmHg的状态下.


为什么有这个差异呢?这也是由于前面的两个原因造成
的,因此,实际气体在压强很大、温度很低时并不严格遵守气
体三定律.

\subsection{麦克斯韦气体分子速率分布}
在热平衡状态下,理想气体分子分布在某一速率间隔
$v$—$v+\Delta v$内的分子数$\Delta N$和该气体的总分子数$N$的比,与速
率$v$有关,与$\Delta v$成正比,即
\[\frac{\Delta N}{N}=f(v)\Delta v\]
对于处在一定温度下的气体,$f(v)$只是速率$v$的函数,叫
做气体分子的速率分布函数.

英国物理学家麦克斯韦,于1859年从理论上确定了气体
分子按速率分布的统计规律,速率分布函数为
\[f(v)=4\pi\left(\frac{m}{2\pi kT}\right)^{3/2}\exp\left(-\frac{mv^2}{2kT}\right)v^2\]
式中$T$是热力学温度,$m$是每个分子的质量,$k$是玻尔兹
曼常数.
\begin{figure}[htp]
    \centering
\includegraphics[scale=.6]{fig/3-13.png}
    \caption{}
\end{figure}


图3.13表示$f(v)$与$v$之间的函数关系的曲线叫做速率
分布曲线.图中纵坐标代表在单位速率间隔的区间内分子数
所占的百分比$f(v)=\dfrac{\Delta N}{N\cdot \Delta v}$
;横坐标表示速率.图中任一区
间$v$—$v+\Delta v$内曲线下的窄条面积,表示气体分子在该区间内
的相对分子数(几率)$f(v)\cdot \Delta v=\frac{\Delta N}{N}$.

由速率分布曲线可见,气体分子的速率可以取由0到$\infty$
之间的一切数值,并且速率很大和很小的分子数较少,所占百
分数较低,而且有中等速率的分子数很多,所占百分率很高,
即所谓“两头小,中间大”的分布规律.

由图中的三速率分布曲线可见,当温度升高时,气体中
速率较小的分子减少,而速率较大的分子增多,曲线的高峰移
向速率大的一方.由于曲线下的总面积应恒为1(归一化条
件),所以曲线变得较平坦.

\subsection{理想气体的压强公式}
大量气体分子对气壁的不断碰撞,就对器壁产生一个持
续的均匀的压力,器壁上单位面积受到的压力就是气体对器
壁的压强,下面我们应用理想气体的分子模型推导气体的压
强公式.

假设有一个边长为$\ell_1$、$\ell_2$及$\ell_3$的长方形容器,容积$V$等
于$\ell_1\ell_2\ell_3$, 其中有$N$个同类气体分子,每个分子的质量都是
$m$. 由于处在平衡状态,容器内分子数目又十分巨大,所以容
器壁上的每个部分都受到大量分子的碰撞,容器中的每个器
壁都受到冲力.由于气体分子在各个方向上运动的几率相
等,所以器壁单位面积上因受到分子的碰撞而产生的压力也
相等.因此要计算气体对器壁的压力,只要计算容器上任何
一个器壁所受的压力就可以了.现在计算与$x$轴垂直的器壁
$A_1$面所受的压力(图3.14).
\begin{figure}[htp]
    \centering
\includegraphics[scale=.6]{fig/3-14.png}
    \caption{}
\end{figure}

先讨论一个速度为$v$的分子,$v$在直角坐标轴上的分量
为$v_x$、$v_y$和$v_z$. 假定分子与器壁的碰撞是弹性的,而且只在
碰撞时,分子与器壁间才有力的作用.当分子和器壁$A_1$面碰
撞时,它受到器壁$A_1$面对它施加的沿-$x$方向的作用力.在
这个力的作用下,分子在$x$轴上的动量由$mv_x$变为$-mv_x$, 分
子在$x$轴上的动量增量为$(-mv_x)-mv_x=-2mv_x$. 根据动
量定理,分子在$x$轴上动量的增量等于器壁给予分子的力的
冲量.根据牛顿第三定律,分子也给予器壁$A_1$面一个大小相
等方向相反的力的冲量,力的方向与$x$轴方向相同.分子对
器壁碰撞的力是间歇的而不是连续的,它从$A_1$面弹回,飞向
$A_2$面,碰撞$A_2$面后,回到$A_1$面再碰撞.分子与$A_1$面作连续
两次碰撞之间,在$x$轴方向上所经过的距离为$2\ell$, 所需要的
时间为$2\ell_1/v_x$. 即在单位时间内,分子与$A_1$面碰撞的次数为
$v_2/2\ell_1$次.上面只讨论了一个分子对器壁碰撞的作用力,这
个力显然只是间歇的打击而不是连续的.但是,实际上容器
中有大量的分子对$A_1$面作连续不断的碰撞,在任何时间内
$A_1$面受到的力可以看作是连续的.这个力的大小应等于单
位时间内全部分子与$A_1$面碰撞所引起的动量增量的总和,即
\[F=2mv_{1x}\cdot \frac{v_{1x}}{2\ell_1}+2mv_{2x}\cdot \frac{v_{2x}}{2\ell_1}+\cdots 2mv_{Nx}\cdot \frac{v_{Nx}}{2\ell_1}\]
式中$v_{1x},v_{2x},\ldots,v_{Nx}$
是各个分子速度在$x$轴上的分量.

$A_1$面受到的压强则为
\begin{equation}
    \begin{split}
 p&=\frac{F}{\ell_2\ell_3}=\frac{m}{\ell_1\ell_2\ell_3}\left(v^2_{1x}+v^2_{2x}+\cdots+v^2_{Nx}\right)\\
 &=\frac{Nm}{\ell_1\ell_2\ell_3} \left(\frac{v^2_{1x}+v^2_{2x}+\cdots+v^2_{Nx}}{N}\right)      
    \end{split}
\end{equation}
式中括弧内的物理量称为分子沿$x$轴方向速度分量的平方的
平均值,用$\overline{v^2_x}$表示.因为
\[\begin{split}
{v}_{1}^{2}&=v_{1 x}^{2}+v_{1 y}^{2}+v_{1 z}^{2} \\
{v}_{2}^{2}&=v_{2 x}^{2}+v_{2 y}^{2}+v_{2 z}^{2} \\
\cdots& \cdots \cdots \\
{v}_{N}^{2}&=v_{N x}^{2}+v_{N y}^{2}+v_{N z}^{2}  
\end{split}\]
所以
\[\frac{v_{1}^{2}+v_{2}^{2}+\cdots+v_{N}^{2}}{N}= \frac{v_{1 x}^{2}+v_{2 x}^{2}+\cdots+v_{N x}^{2}}{N}+\frac{v_{1 y}^{2}+v_{2 y}^{2}+\cdots+v_{N y}^{2}}{N} +\frac{v_{1 z}^{2}+v_{2 z}^{2}+\cdots+v_{N z}^{2}}{N}\]
上式右边三项各表示沿$x$、$y$、$z$三个轴的速度分量的平方的
平均值$\overline{v^2_x},\overline{v^2_y},\overline{v^2_z}$,左边一项则表示所有分子速度的平方的平
均值$\overline{v^2}$, 所以
\[\overline{v^2}=\overline{v^2_x}+\overline{v^2_y}+\overline{v^2_z}\]
设$\overline{v^2_x}=\overline{v^2_y}=\overline{v^2_z}=\dfrac{1}{3}\overline{v^2}$,代入(3.3)式,并设$n=\dfrac{N}{\ell_1\ell_2\ell_3}$为单位体积内的分子数,则(3.3)式为:
\[p=\frac{1}{3}nm\overline{v^2}\]
或
\begin{equation}
    p=\frac{2}{3}n\left(\frac{1}{2}m\overline{v^2}\right)
\end{equation}
上式叫做理想气体的压强公式.由上式可知,气体作用于器
壁的压强和单位体积内的分子数$n$成正比,和分子的平均平动动能
$\frac{1}{2}m\overline{v^2}$成正比.这说明了气体压强这个宏观量的微观
本质.气体的压强是大量分子与器壁碰撞所产生的平均效
果,它和分子的平均平动动能的关系是一个统计性的规律,所
以离开了“大量分子”和“统计平均”来谈气体压强是没有意义
的.换句话说,气体压强只具有统计的意义.



\subsection{绝对零度及人类向低温的进军}
$-273.15^{\circ}{\rm C}$作为热力学温标(绝对温标)的零度,叫做绝
对零度.

绝对零度是根据理想气体所遵循的规律,用外推的方法
得到的.用此方法推出,在绝对零度时,气体的体积将减小到
零,理想气体分子将停止运动,这些都是理想的推理.事实上
一切实际气体在温度接近绝对零度前,早已变成液态或固态,
分子运动已不再遵从经典物理的热力学统计规律,而表现出
明显的量子特征.通过大量实验以及经过量子力学修正后的
理论导出,在接近绝对零度的地方,分子的动能趋于一个固定
值,这个值叫做零点能量.这说明绝对零度时,分子的能量并
不为零,而是具有一个很小的数值.原因是,全部粒子都处于
能量可能有的最低的状态,也就是全部粒子都处于基态.

绝对零度是低温的极限,只能无限逼近,不能达到.

19世纪20年代,法拉第首先发现,在适当的温度下,给
某些气体施加足够大的压力,就可以使它们变为液体.而
这些液体又是一种极好的冷却剂,因为当它们在减压条件下
蒸发而变成气体的时候,会从周围环境吸收热量,使温度降得
更低.因此,利用气体的液化取得低温成为当时的重要课
题.经过10多年的努力,物理学家获得了$-110^{\circ}{\rm C}$的低温,但
少数气体比如氢、氧、氮、一氧化碳及氦等,无论施加多大压
力,也不能液化.

1869年爱尔兰物理学家安德鲁斯发现,任何气体都有一
个临界温度,高于这个温度,无论施加多大压力,也不能
液化.

随后,德国科学家林德采用压缩—绝热膨胀法和抽除液面
蒸气法终于制得氧气和氮气的液滴.

1898年,苏格兰化学家杜瓦,在$-253^{\circ}{\rm C}$的低温下液化了
氢气.一年后得到固态氢,达到10—12K的低温,荷兰物理
学家翁内斯用毕生精力,采用综合法,使氦气冷却到4.2K, 这
样才首次得到了液体氦.这是地球上最后被液化的一种气体,
时间是在1908年.

之后,科学家们寻求别的办法取得低温,如荷兰物理学家
德拜找到了一种获得超低温的新方法——绝热去磁法,德国
物理学家伦敦发明了稀释致冷法,1957年创造了0.00002K
的低温记录,现在,人们已经获得了0.0000001K的最低温度.
不管物体的温度降得多么低,绝对零度是不可能达到的.人类
在向绝对零度进军的过程中创造了一系列的致冷技术;加深
了对物质本质的认识.(摘自《物理教学》1981年第2期,朱
志尧:《向绝对零度进军》)





