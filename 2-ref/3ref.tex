
\chapter{气体的性质}
\section{教学要求}
从实验出发,建立气体实验定律,对实验定律进行研究而获得理想气体的状态方程,最后从微观上利用气体分子运动论和统计方法解释气体实验定律,本章教材的这种安排,较鲜明地体现了实验的作用和理论的指导意义,体现了宏观和微观的统一,这是本章教材的特点,注意到这一点,无疑对学生有科学研究方法上的启迪。

本章可分为四个单元,第一单元由第一节至第四节,从介绍气体的三个状态参量开始,研究气体的两个实验定律并引出热力学温标,第二单元包括五、六两节,主要研究理想气体的状态方程和克拉珀龙方程,第三单元由七、八两节组成,介绍气体分子运动的特点并对气体的实验定律进行定性的微观解释,第四单元为九、十两节,从能量的角度研究理想气体并讨论了理想气体在几个等值过程中内能的变化。

气体的三个状态参量,体积、压强、温度是研究气体性质的基础,在教学中,要求学生理解这三个状态参量的意义,由于压强的计算稍难一点,因此在教学中要多花一些力气搞清
压强的计算和单位间的关系。

气体三个实验定律反映了气体在温度不太低、压强不太高的条件下状态变化所遵循的规律,是建立理想气体状态方程的依据,因此,要求学生很好的掌握,对气体等温变化的图线和气体等容变化的图线的物理意义要清楚地理解。

教材是利用气体等容变化的$p$-$t$图线交于${\rm -273^{\circ}C}$引出绝对零度从而建立热力学温标的,热力学温标的建立对简化查理定律并导出理想气体状态方程起了重要作用 因此,要使学生掌握热力学温度和摄氏温度的换算关系,理解热力学温标和绝对零度的意义。

理想气体的状态方程反映了一定质量的理想气体三个状态参量间的变化关系,是本章的重点知识,对于进一步学习和生产实践都有重要的意义,要求学生巩固掌握,克拉珀龙方程是本章的重要知识,但不是重点,要求学生掌握,但不要去做过多过深的习题,以免加重学生的负担。

关于气体分子运动的特点,以及对气体实验定律的微观解释,教材在介绍知识的同时,渗透一些统计的观点,由于这部分内容比较抽象,在中学也不可能讲解透彻,因此,不宜讲得过多过深,不宜要求过高。

理想气体的内能一节介绍了理想气体的微观模型,指出了它的内能只跟温度有关,跟气体体积无关,了解了理想气体的微观模型,就能理解为什么实际气体在温度不太低,压强不太高时,其性质跟理想气体接近,并为进一步从能量角度研究理想气体打下基础。关于理想气体内能的变化,应根据学生情况决定讲与不讲,讲到什么程度。这一节内容,对热力学第
一定律用于理想气体的几个等值过程做了粗略的介绍,利于
扩大学生的知识面,但限于学生实际水平,只能要求他们定性地了解这些知识。

这一章的教学要求是:
\begin{enumerate}
 \item 理解描述气体状态的三个参量(体积、压强、温度)的意义,会计算气体的压强。
  \item 掌握玻意耳-马略特定律、查理定律、盖·吕萨克定律,理解气体等温变化的p-V图象和气体等容变化的p-t、p-T图象的意义,会用这三个定律计算有关的问题。
  \item 理解热力学温标和绝对零度的意义,掌握热力学温度和摄氏温度的换算关系。
  \item 巩固掌握理想气体的状态方程,掌握克拉珀龙方程,会用它们来计算有关问题。
  \item 了解气体分子运动的特点,了解气体压强的微观解释,会用气体分子运动论对气体的三个实验定律进行微观解释。
  \item 了解理想气体的微观模型,了解理想气体的内能只跟温度有关,跟体积无关。了解理想气体在几种等值过程中内能的变化情况。
\end{enumerate}
 
\section{教学建议}
\subsection{第一单元}
\subsubsection{气体的状态和状态参量}

本节研究了描述气体状态的三个参量,提出了本章研究的中心课题,即确定三个状态参
量之间的变化规律,指出了研究方法:保持其中一个量不变研究其他两个量的关系,然后再确定三个状态参量之间的关系。让学生在学习本节内容时,明确以上各点能够增强学生学习的目的性。

在介绍描述气体状态的状态参量时,应使学生明确地认识到,在物理学中因研究的问题不同,所用的物理量也不同。在热学中研究气体的性质,所用参量有体积、压强和温度。

在讲解本节内容时,应注意三点:
\begin{enumerate}
\item 气体体积是指气体分子充满的空间,即容器的容积.这个容积不是气体分子本身的体积之和,气体分子之间是有空隙的。
\item 对气体压强的计算方法,单位换算要给予足够的重视。因为它是后面学习气体定律和气态方程的基础,正确地确定气体的压强往往是应用气态方程解决问题的关键,恰当地选择压强的单位可以避免繁杂的计算。在教学中要多花些力气,学生也应做一定练习,务求掌握气体压强的计算方法和单位之间的换算关系。
\item 教材中提到“对一定质量的气体来说,如果体积、压强和温度这三个量都不改变,我们就说气体处于一定的状态中”。这个“状态”指的是气体的热动平衡态,有关热动平衡态,在课堂上无需涉及,在这里,只要求学生理解上面那段话的意思就行了。
\end{enumerate}

\subsubsection{气体的等温变化}
 玻意耳-马略特定律 节利用实验研究一定质量气体的等温变化,总结出玻-马定律,对等温线作了介绍,最后指明了玻-马定律的物理意义和适用
 范围。

 讲好玻-马定律的关键是做好演示实验,用教材中介绍的装置,也可以验证在温度不变的条件下,$pV=$恒量。

 在做演示实验前,要引导学生仔细观察实验装置,明确指出研究对象是$A$管中密闭的气体,提出下列问题让学生思考回答:
 \begin{enumerate}
 \item 如教材39页图3.5所示,$A$管中气体的压强的表达式是怎样的?
 \item $A$管中的气体体积怎样确定?
 \item 为了保证实验中气体恒温,$B$管的升降快慢有什么要求?为什么?实验中的恒温是什么温度?
 \end{enumerate}

 该实验不一定采用书上介绍的装置,也可用其他仪器进行,不论采用哪种办法,其共同点是保证气体的定质量和等温变化的条件。 

 对于等温线,主要讲清它表明了一定质量气体在等温条件下压强和体积的变化关系,不要求利用图线进行定量计算。教材中给出了不同温度下的几条等温线,主要是说明$pV=$恒量这个公式中的恒量跟温度的关系,得出对于一定质量的气体,“恒量随温度的增高而增大”的结论。显然这个恒量不是一个普适恒量。因此,要得出$p$、$V$、$T$三个状态参量之间的关系,还必须研究$P$、$T$或$V$、$T$之间的规律。关于这一点,教师可对比牛顿第二定律的研究过程:在$F$、$m$、$a$三个物理量中先研究加速度和力的关系,再研究加速度和质量的关系,从而确立了三个量之间的关系,引导学生注意这种控制条件的研究方法。

 教材选用了一个漏气的例题,这是一个有利于培养学生分析能力的题目,应该引导学生分析讨论,由于该题是第一
 个利用玻-马定律求解的题目,教师一定要明确指出,确定研究的对象是哪一定量的气体,对它进行变化前后的状态分析是正确解题的前提,教材也为此做了很好的示范,教师要充分注意这个问题,使学生一开始就养成良好的习惯、掌握正确的分析方法,由于本章习题涉及的物理量多,状态往往在两个或两个以上,做到这一点,尤为重要。

\subsubsection{气体的等容变化~~查理定律}

本节利用实验定性地归纳出在体积不变的条件下,一定质量的气体的压强随温度的升高而增大,随温度的降低而减小,然后给查理定律的表述和公式,给出气体等容变化的图线,最后明确查理定律的适用范围。

 教学中应对查理定律演示实验进行精心准备,使实验达到定量的效果,用实验所取得的数据作出如教材第47页的$p$-$t$图线,其延长线在实验误差范围内交于$-273^{\circ}{\rm C}$. 这样做的好处是:
 \begin{enumerate}
     \item 能突出物理规律的实验研究方法;
     \item 由实验图线获得$-273^{\circ}{\rm C}$的外推点,为下节研究热力学温标打下基础;
     \item 查理定律可以由图象总结出来。
 \end{enumerate}

 由于学生已学过斜截式的直线方程,比较容易总结出查理定律的教学表达式:由图线写出方程
\[p=kt+p_0\]
其中$p_0$为$0^{\circ}{\rm C}$时气体的压强,斜率$K=\frac{p_0}{273}$,
所以
\[p=\frac{p_0}{273}\cdot t+p_0\]
即
\[p=p_0\left(1+\frac{t}{273}\right)\]

为了能定量地进行实验,一要用气压表测定实验时的大气压;二要在$0^{\circ}{\rm C}$以下得出一组$p$、$t$数据,才能使$p$-$t$图象较准确地交$t$轴于${\rm -273^{\circ}C}$。为此,可用食盐加碎冰溶化获得零下20多摄氏度的低温;三要注意给出充分地时间让气体跟周围环境达到热平衡。

同讲解玻-马定律一样,教材在讲解查理定律后又强调了定律的适用范围是压强不太大,温度不太低,这点仍须向学生明确指明,物理定律都有它的适用范围。超出了这个范围,物理现象所遵从的规律将发生变化,不再是原来的规律了。对于气体定律的适用范围,学过理想气体的微观模型以后,就可以进一步理解它的道理了。

\subsubsection{热力学温标}

由查理定律外推而引入新的温标,绝不意味着查理定律的适用范围可以“外推”,在低温下,查理定律不适用了,这点要再次强调。

“外推点”定为热力学温标的零度,也绝不是一次实验的结果,而是对不同气体做多次实验所作出的图线都交$t$轴于${\rm -273^{\circ}C}$, 这才赋予它以新的物理意义,引出绝对零度,这一点在教学中要交代清楚。

在引出热力学温标后,要强调“就每一度的大小来说,热力学温度和摄氏温度是相同的”。这就是说,虽然热力学温度跟摄氏温度间的关系为$T=t+273.15$(K), 但就\textbf{同一温度变化}来看,利用热力学温标与利用摄氏温标所表示的数值是相同的,即$\Delta T=\Delta t$. 这一点在计算题中经常用到。

由实验总结出查理定律,又由查理定律“外推”到零压强而引入热力学温标、绝对零度,这一方面体现了实验的重要
性,另方面也体现了理论的指导作用,正是由于这种理论上的外推使人们获得了绝对零度的概念,导致了人类不断向绝对零度进军的历史进程,向学生介绍这点可以启迪思维,加深对物理研究方法的认识,结合讲解一些人类向绝对零度进军的情况,可以大大提高学生兴趣,开阔学生眼界。

\subsection{第二单元}
\subsubsection{理想气体的状态方程}

本节利用玻-马定律和查理定律推出一定质量气体的状态方程,提出了理想气体这一概念,说明了状态方程包含了玻-马定律和查理定律,并由它推出了另一个实验定律——盖·吕萨克定律。

对盖·吕萨克定律,教材中叙述得比较简单,教师可以补做演示,使学生的印象深刻些。

要再次强调状态方程的适用范围,强调理想气体是一种科学的抽象,一个理想的物理模型,要使学生明确地认识到,理想气体虽然并不存在,但它是实际气体在一定程度上的近似,在压强不太大,温度不太高的条件下可把实际气体作为理想气体来处理。

课后布置教材习题4, 根据玻-马定律和盖·吕萨克定律推出一定质量的理想气体的状态方程$pV/T=$恒量,向学生指明,只要初末状态确定,不论经过哪两个等值过程,都可以推出状态方程。这种练习,对于培养学生发散性思维是有帮助的。

\subsubsection{克拉珀龙方程}

克拉珀龙方程是任意质量的理想气体的状态方程,推导克拉珀龙方程时,要注意启发性,注意启
发的层次。教材是按下列层次安排的:
\begin{enumerate}
\item 从$pV/T=$恒量开始讨论,说明这个恒量与气体的质量和气体的种类有关;
\item 研究1摩尔任何气体的$p_0V_0/T_0$值,即摩尔气体恒量$R$, 从而得到1摩尔理想气体的状态方程$pV=RT$;
\item 由标准状态下,$n$摩尔气体占有体积为$nV_0$, 利用摩尔气体恒量$R$导出任意质量的理想气体的状态方程,即克拉珀龙方程$pV=\frac{m}{M}RT$.
\item 最后说明气体满足方程的条件。
\end{enumerate}

教材这样安排,层次是清楚的,启发性也较强。在讲解中,要指出摩尔气体恒量是热学中的一个重要常数,限于学生知识水平,其重要性不可能在课堂上分析,但值得注意的是,在讲解中不要把$R=p_0V_0/T_0$误认为是$R$的物理意义,上式仅仅是$R$的计算方法,由于在标准状态下,1摩尔的任何气体体积$V_0$都是22.4升,上式表明$R$是一个普适恒量,$R$的物理意义是:1摩尔理想气体在压强不变时,温度升高1开对外界所做的功。R的物理意义在讲授中不宜提出,更不能给出错误的解释,这一点应该注意。

在应用克拉珀龙方程解题前,还必须指出,由于$R=8.31$焦/摩·开$=0.082$标准大气压·升/摩·开,因此,$p$、$V$的单位必须与选用的$R$的单位相适应。

利用三个实验定律和两个状态方程解题时,要注意训练学生对气体变化前后的状态进行分析,教师要做出表率。

\subsection{第三单元}
本单元通过对气体分子运动的特点的分析,对气体实验定律进行微观解释,由于本单元的内容较为抽象,因此,一方面要严格控制,不要增加教材以外的内容;另一方面要使学生对分子间距离的大小和分子间的频繁碰撞在头脑中有深刻的、形象的认识。

讲解气体分子运动的特点要渗透一些统计的观点,要使学生知道,个别分子的运动具有偶然性,但大量分子在总体上都服从确定的规律,这种规律是统计性的,叫做统计规律。

为了讲好“气体实验定律的微观解释”,建议在讲课开始时,复习压强的概念、公式及弹性小球对墙壁的碰撞给与墙壁的冲量的计算,为正课的讲授扫清一些障碍。

压强的微观解释是本节的重点,在讲解这个问题时,要使学生体会到,虽然个别分子的运动服从力学规律,个别分子碰撞器壁时所产生的冲力也要用力学规律来计算,但涉及大量分子的集体行为却要用到统计观点,因此,不是一种纯粹的力学问题。

另外,应注意形成气体压强的微观图景,教材是以雨滴对伞的压强为例来说明的,为了加深理解,建议课堂上进行模拟气体压强的演示实验,以增加学生的实感。

\subsection{第四单元}
本单元的必学教材为第九节,该节主要建立理想气体的
微观模型,并指出理想气体的内能就是气体所有分子热运动的动能的总合,不存在分子势能,因此理想气体的内能仅与温度有关而与体积无关,建议在讲解本节内容前,认真复习第二章第一节“物体的内能”,让学生搞清分子动能、分子势能是由什么因素决定的,在此基础上,本节内容是容易理解的。

第十节是选学教材,对于基础较好的班可以选用,本节研究各种等值过程中理想气体的内能的变化,其研究的手段是利用热力学第一定律,该节的全部内容可列为下表。

\begin{table}[htp]
\caption{理想气体等值过程的讨论}
\begin{tabular}{cp{.32\textwidth}p{.18\textwidth}c}
\hline
过程名称& 各等值过程的物理现象 & 热力学第一定律的形式&$p$-$V$图\\
    \hline
等温过程&内能不变$\Delta E=0$,\par 吸热则体积膨胀,对外做功;放热则体积缩小,外界对气体做功
&$W+Q=0$&\raisebox{-.65\height}{\tikz[>=latex, yshift=-3cm]
 {\draw[<->](0,2.5)node[right]{$p$}--(0,0)--(2.5,0)node[above]{$V$};\draw(.34,2) [bend left=-35] to node[right]{等温线} (2,.34);
 }        }
 \\
\hline
等容过程&气体不做功$W=0$,\par 吸热则内能增加;放热则内能减少
&$Q=\Delta E$&\raisebox{-.45\height}{\tikz[>=latex]
 {\draw[<->](0,2.5)node[right]{$p$}--(0,0)--(2.5,0)node[above]{$V$};\draw(1,0) to node[right]{等容线} (1,2.5);
 }        }
\\
\hline
等压过程&膨胀时,气体对外做功,其密度减小,为保持压强不变,必须吸热升温,故内能增加;压缩时,情况相反
&$W+Q=\Delta E$&\raisebox{-.85\height}{\tikz[>=latex]
 {\draw[<->](0,2.5)node[right]{$p$}--(0,0)--(2.5,0)node[above]{$V$};\draw(0,1) to node[above]{等压线} (2.5,1);
 }        }
\\
\hline
绝热过程&气体与外界无热交换,$Q=0$.
体积膨胀,对外做功,内能减少;体积压缩,外界对气体做功,内能增加
&$W=\Delta E$&\raisebox{-.85\height}{\tikz[>=latex]
 {\draw[<->](0,2.5)node[right]{$p$}--(0,0)--(2.5,0)node[above]{$V$};\draw(.34,2) [bend left=-35] to node[right]{绝热线} (2,.34);
 }        }
\\
\hline
\end{tabular}
\end{table}





