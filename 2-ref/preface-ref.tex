\chapter{高中物理甲种本第二册的说明}

(1)高中物理课本(甲种本)第二册讲述的有热学和电学
两部分内容,热学共分五章。第一章讲述分子运动论的基本
内容。第二章讲述内能、热力学第一定律以及能的转化和守
恒定律,研究热现象有两种方法,即微观方法和宏观方法,它
们相辅相成使热学的研究越来越深入。与此相适应,教材头
两章分别介绍这两种方法,给整个热学知识的讲解打好基础,
因此这两章是基础性的两章。这两章所讲的知识要贯穿在整
个热学知识中。其中,能的转化和守恒定律是自然界的普遍
规律,不仅贯穿热学部分,而且要贯穿以后所学的知识。

气体比较简单,热学性质研究得比较透彻,中学阶段有可
能讲得扎实深入一些,因此教材把第三章“气体的性质”作为
热学部分的重点知识来讲解。着重讲解了理想气体状态方
程,用气体分子运动论解释了气体的三个实验定律。最后介
绍理想气体的内能,并综合运用热力学第一定律和气体性质
的知识分析了理想气体内能的变化。第四章介绍液体和固体
的性质。这一章是介绍性的,从宏观上介绍了晶体的性质和
液体的表面现象,并从微观上作适当解释,把宏观现象微观
解释联系起来,第五章讲述物态变化,这部分知识在许多方
面都要用到,初中虽然学过一些,但讲得比较简单,高中要予
以扩展,并给予必要的微观解释,以便使学生对知识的理解深
化一些。

这一册教材的后一部分讲述三章电学知识。第六章讲静
电学,讲述电学中的几个基本物理量(场强、电势、电容等),并
初步接触到物理学中一个基本概念一场。这些知识很重
要,是学习电学知识的基础,因此这一章是电学部分基础性的
重要一章,第七章讲述直流电路方面的知识,既是重要的基
础知识,以后学习中常常用到,而且具有实际意义。第八章讲
述物质的导电性,包括金属的导电性,液体和气体的导电性,
真空中的电流以及半导体的导电性。学习电学知识,了解各
种物质的导电机理,有利于加深学生对导电现象和导电规律
的理解,有利于知识的系统化。

(2)这册教材的内容同第一册力学比较起来,定量讲解
的内容相对减少,定性分析物理现象的内容相对增加,这样
做是必要的。中学物理教学不但要有一定的深度,而且要有
一定的广度,不但要按部就班系统地学习知识,而且要灵活
渗透一些知识。学知识面窄,眼界不宽,兴趣不广,思路不
灵活,这对在中学学好物理不利,对将来参加工作和进一步学
习也不利。物理教学不能仅仅是物理计算,而且要重视引导
学生灵活地运用概念和规律分析解释物理现象。即使物理计
算,也不能刻板地去搞所谓类型,而应着重于物理思考,学习
分析问题的方法,培养分析物理问题的能力,不能因为某部
分教材的定量计算少,就忽视这部分教材。各部分教材都有
它本身的作用,不能忽视。

这册教材涉及到微观领域的知识。教材在讲热学知识时
注意用微观知识解释热现象及其规律,在讲解物质的导电性
时注意说明导电机理 虽然有些知识,学生在初中物理课中
学过,或者在化学课中学过,但是人类是怎样获取这些知识
的,学生并不很了解。微观世界看不见,摸不到,而人们却能
掌握微观世界的规律。人类进入微观世界并不是凭空想象出
来的,而是经过实验、经过分析,建立各种假设和模型,并进而
经过实验的检验,逐步认识微观世界的,作为物理课,讲解微
观领域的知识,要着重讲述这些知识的实验基础,注意使学生
知道人们进行微观世界的线索和思路,人们认识了微观世
界,反过来又对宏观现象进行微观解释,从而对宏观现象的认
识更深入一步,因此,培养学生用微观世界的知识对宏观现
象进行微观解释,在物理教学中是十分重要的。

这册教材较多的涉及到一些物理恒量,如摩尔气体恒量、
阿伏伽德罗常数、基本电荷、法拉第恒量等。物理恒量及其测
定,在物理学的发展中起重要作用。物理恒量的存在是物理
世界的规律性的反映,用不同的方法测得的物理恒量的数值
相符,表明物理学理论本身和谐一致,也证明理论正确地反映
了客观规律 因此,在高中物理教学中,适当地讲解一些恒量
及其测定是很有意义的,它可以开阔学生的思路,培养他们
灵活地运用知识,对于微观恒量,要使学生知道它们的大小
都是通过宏观量测定而得来的,这有助于使学生了解人类进
入微观世界的线索。

这册教材进一步扩展和加强了能的转化和守恒的观点。
在静电学中对电势能的讲解比较仔细,使学生对功和能之间
的关系的认识得到扩展,通过分析电荷在电场中的运动,动
能定理扩展到电场力做功的情形;同时,能的转化扩展到电势
能与运动电荷的动能之间的转化,在稳恒电流一章中,继续注
意贯穿能的转化和守恒这条线索 讲解电功时,明确指出电流
做功实际上是电场力移动电荷做功,因而电流做功的过实
质上是电能转化为其他形式的能的过程。为了使学生搞清电
路中能的转化关系,教材指出:在纯电阻电路中,电完全转
化为内能;在非纯电阻电路中,电能除转化为内能外,还转化
为其他形式的能,如在电动机中转化为机械能,在这种转化
中能量守恒。关于电动势的讲法,虽然没有讲非静电力做功,
但仍注意从能的转化角度进行分析 这样,讲过闭合电路的
欧姆定律之后,学生对整个电路中能的转化情况将会有一个
全面了解,并能够从能的转化的观点分析有关电路的问题,建
议在这册教材的总复习时,对力学中讲过的功能关系予以扩
展,对能的转化和守恒作一个小结,有意识地加强培养学生用
能的观点分析问题的能力。

(3)这册教材在讲解知识时,注意了渗透研究物理的方
法,培养分析问题的能力。

在物理学中常常用各种模型来反映客观事物。通过建立
模型来进行研究,这是研究物理学的重要方法。在高中物理
中要讲到一些模型。力学中的质点,其实就是一种简单的模
型,这册教材在讲解分子运动论时,涉及分子的模型。要使
学生理解,因问题的需要,对分子可以采取不同的模型。模型
常常是对现实的一种简化或理想化,至于哪些因素被忽略,要
看问题的性质而定,这册教材还提到了理想气体的微观模型。
要使学生知道建立模型的重要,并且逐渐习惯于运用模型来
分析和考虑问题。同时,也要适当指出模型的近似性,不要把
模型当成是绝对完整地反映实际事物的 这可以发展他们的
想象能力和思维能力,培养他们灵活地运用知识,在教学中
值得重视。
在物理学中常常要利用两个量的比值定义一个新的物理
量。从第一册开始,速度就是用比值来定义的。这册教材在
讲电场强度、电势等概念时也是用比值来定义的,为了使学
生对用比值定义物理量这种方法进一步熟悉起来,教材安排
了一个阅读材料,对这个问题加以说明,物理量的定义方法
是各种各样的,除了基本的物理量而外,总是要通过已知的物
理量来定义新的物理量。用比值定义物理量只是一种形式,当
然还有其他形式。功、动量、冲量就是用两个量的乘积来定义
的,建议指导学生学习这个阅读材料,并对如何定义物理量,
作些指导与说明,这将有助于学生今后学好物理知识。

“等效”的概念在分析问题时常常用到,这册教材在讲述
串并联电路时提出了等效电阻的概念,目的是要使学生初步
接触一下等效的概念。教师如果感到必要,也可以提前,在讲
述电容的串并联时提出。对某些问题可以介绍学生用等效的
方法来处理。比如,一个复杂的电路,只要知道这个电路两端
的电压和通入(或流出)的电流,根据等效电阻的概念,就可以
直接得出这个电路的总电阻,学会用这种方法来处理问题,对
于学生灵活运用知识是有好处的。

利用图线可以形象地表达物理规律。学生熟悉图线的表
示方法,将有助于他们掌握物理规律,对规律所表达的内容有
一个形象具体的理解。这册教材对图线的利用有所加强。教
材在讲述分子的互相作用、气体的实验定律、欧姆定律、串联
电路的电压、路端电压等问题时,都利用了图线。对图线的教
学,最基本的是使学生清楚地理解图线所表达的物理意义,看
到图线就能在头脑中对物理的变化规律有具体的认识。为了
减轻学生的负担,教材并不要求利用图线进行定量的计算。

(4)这册教材一共安排了十三个学生实验.这些实验,
从实验训练角度上看,所涉及的方面还是比较全的。就实验
的性质来说,有基本定律的验证,有物理量的测定,也有新的
仪器的使用,就使用的仪器来说,中学阶段要求掌握的仪器,
大部分使用到了、因此这册教材的学生实验,在整个中学的
实验教学中具有重要的地位和作用。

在少实验中,要用到初中和高一已经学过的仪器,学生
对这些仪器的掌握情况,直接关系到这册中的实验效果,因
此,教学中要特别注意提高学生掌握和使用这些仪器的
技能。

学生实验同第一册一样,要求学生在理解实验原理的基
础上知道怎样操作以及为什么这样做,避免依照书本上的步
骤盲目操作。“测定金属的电阻率”这个实验,原理比较简单,
要求学生自己选择实验仪器和确定实验步骤,这样做,要求提
高了,教学中可先让学生讨论,或给以必要的提示,“练习使用
万用表”这个实验,没有提出具体测量项目。这是为了便于教
师在教学中自己掌握,根据学校器材情况来确定测量项目。
(5)高中物理(甲种本)第二册的教学内容可按每周3课
时,全学年共96课时讲授完。各章所用的课时数是:第一章
分子运动论基础5课时,第二章内能、能的转化和守恒定律
6课时,第三章气体的性质13(2)课时(括号内的数字是学生
实验的课时数,下同),第四章固体和液体的性质5课时,第五
章物态变化10课时,第六章电场17(2)课时,第七章稳恒电
流18(7)课时,第八章物质的导电性14(4)课时,平时复习
和机动时间8课时。




































